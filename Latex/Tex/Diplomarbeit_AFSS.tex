\documentclass{article}
\usepackage[utf8]{inputenc}  % Ensure this is loaded before biblatex
\usepackage{biblatex}  % Load the biblatex package
\addbibresource{./../Quellen/sources.bib}  % Add your .bib file


%%%%%%%%%%%%%%%%%%%%%%%%%%%%% Using Packages %%%%%%%%%%%%%%%%%%%%%%%%%%%%%%%%%%
\usepackage[a4paper, margin=2cm, top=5mm, includehead, headheight=2cm]{geometry}
\usepackage{graphicx}
\usepackage{amssymb}
\usepackage{amsmath}
\usepackage{amsthm}
\usepackage{empheq}
\usepackage{mdframed}
\usepackage{booktabs}
\usepackage{lipsum}
\usepackage{graphicx}
\usepackage{color}
\usepackage{psfrag}
\usepackage{pgfplots}
\usepackage{bm}
\usepackage{float}
\usepackage{wrapfig}
\usepackage{enumitem}
\usepackage[export]{adjustbox} % for valign option
\usepackage{caption}
\usepackage{subcaption}
\usepackage{hyperref}
\usepackage{array}
\usepackage{tocloft}
\usepackage[section]{placeins}
\usepackage[nottoc,numbib]{tocbibind}


\usepackage{emptypage}
\usepackage{fancyhdr}

\usepackage{helvet}

\renewcommand{\familydefault}{\sfdefault}

\renewcommand{\footrulewidth}{0.4pt}%
\renewcommand{\headrulewidth}{0.4pt}%

\renewcommand{\headruleskip}{3mm}
\renewcommand{\footruleskip}{4mm}

\usepackage[english,ngerman]{babel}
\usepackage[ddmmyyyy]{datetime}

\newcommand{\todayD}{\the\day.\the\month.\the\year}   

\graphicspath{{../Bilder/}}

\setlength\parindent{0pt}
\counterwithin{figure}{section}
\counterwithin{table}{section}



%%%%%%%%%%%%%%%%%%%%%%%%%%%%%%%%%%%%%%%%%%%%%%%%%%%%%%%%%%%%%%%%%%%%%%%%%%%%%%%

% Other Settings
\usepackage[dvipsnames]{xcolor}

%\pagecolor[rgb]{0,0,0} %black

%\color[rgb]{0.5,0.5,0.5} %grey

%uncomment to make the links not display the red border
%\hypersetup{
%    colorlinks,
%    citecolor=black,
%    filecolor=black,
%    linkcolor=black,
%    urlcolor=black
%}


%% codestuff
\usepackage{listings, listings-rust}
\setlength{\parindent}{8pt}
\usepackage{indentfirst}
\definecolor{codegreen}{rgb}{0,0.6,0}
\definecolor{codegray}{rgb}{0.5,0.5,0.5}
\definecolor{codecomment}{rgb}{0.6,0.6,0.6}
\definecolor{backcolour}{rgb}{0.92,0.92,0.91}


\lstdefinestyle{mystyle}{
    backgroundcolor=\color{backcolour},   
    commentstyle=\color{codecomment},
    keywordstyle=\color{NavyBlue},
    numberstyle=\tiny\color{codegray},
    stringstyle=\color{codepurple},
    basicstyle=\ttfamily\footnotesize\bfseries,
    breakatwhitespace=false,         
    %breaklines=true,                 
    captionpos=t,                    
    keepspaces=true,                 
    %numbers=left,
    numbersep=5pt,                  
    showspaces=false,                
    showstringspaces=false,
    showtabs=false,                  
    tabsize=2
}
% -- Setting up the custom style:
\lstset{style=mystyle}

%%%%%%%%%%%%%%%%%%%%%%%%%% Page Setting %%%%%%%%%%%%%%%%%%%%%%%%%%%%%%%%%%%%%%%
\geometry{a4paper}

\newcommand{\iu}{{i\mkern1mu}}
\newcommand*\mathinhead[2]{\texorpdfstring{$\boldsymbol{#1}$}{#2}}


\definecolor{mred}{rgb}{0.619, 0.2392, 0.3176}


%%%%%%%%%%%%%%%%%%%%%%%%%%%%%%% Title & Author %%%%%%%%%%%%%%%%%%%%%%%%%%%%%%%%
\title{AFSS - Automated Factory Storage System}
\author{Benedikt Simbürger \\
    \and Nikolaj Voglauer der allerechte \\
    \and Vincent Sonvilla \\
    \and Elena Widmann
    }

%%%%%%%%%%%%%%%%%%%%%%%%%%%%%%%%%%%%%%%%%%%%%%%%%%%%%%%%%%%%%%%%%%%%%%%%%%%%%%%

\begin{document}
\pagenumbering{gobble}

\begin{figure}[h]
    \includegraphics[width=0.5\textwidth]{HTL_Moessingerstraßen_Logo.png}
    \centering
\end{figure}

\begin{center}
    \huge \textbf{HÖHERE TECHNISCHE BUNDESLEHRANSTALT} \\
    \vspace{5mm}
    \Large{KLAGENFURT, MÖSSINGERSTRASSE}

\end{center}

\vspace{7mm}

\begin{center}
    \Large{ABTEILUNG ELEKTROTECHNIK}
\end{center}

\hrule

\vspace{10mm}

\begin{center}
    \Huge \textbf{DIPLOMARBEIT} \\
    \vspace{7mm}
    \huge{Titel der Diplomarbeit Deutsch}

    \vspace{7mm}
    \huge{Automated-Factory-Storage-System}

    \vspace{7mm}
    \Large{JAHRGANG 5AHET}

\end{center}

\vspace{20mm}

\begin{flushleft}
    \bgroup
        \Large
        \def\arraystretch{1.5}
        \begin{tabular}{p{5cm}l}
            eingereicht von & Benedikt Simbürger\\
            & Vincent Sonvilla\\
            & Nikolaj Voglauer\\
            & Elena Widmann\\
            Projektbetreuer & Dipl.-Ing. Christian Sallinger
        \end{tabular}
    \egroup
\end{flushleft}

\vspace{7mm}
\Large
Diese Diplomarbeit entspricht den Standards gemäß dem Leitfaden zur Umsetzung der Reife- und Diplomprüfung des BMBWF in der letztgültigen Fassung.\par
\begin{flushright}
    Klagenfurt, am 04.04.2025
\end{flushright}

\newpage

\begin{figure}[h]
    \includegraphics[width=0.5\textwidth]{HTL_Moessingerstraßen_Logo.png}
    \centering
\end{figure}

\begin{center}
    \huge \textbf{EIDESSTATTLICHE ERKLÄRUNG}\\
    \vspace{7mm}
    \Large
    \begin{tabular}{p{14cm}}
        Ich versichere an Eides statt, dass ich diese Diplomarbeit
        selbstständig verfasst und keine anderen als die angegebenen
        Quellen und Hilfsmittel verwendet habe. Alle Gedanken, die im
        Wortlaut oder in grundlegenden Inhalten aus unveröffentlichten
        Texten oder aus veröffentlichter Literatur übernommen, oder mit
        künstlicher Intelligenz generiert wurden, sind ordnungsgemäß
        gekennzeichnet, zitiert und mit genauer Quellenangabe
        versehen.
    \end{tabular}

    \vspace{10mm}
    Verfasser/Verfasserin\\
    
    \vspace{40mm}
    \bgroup
        \newcolumntype{P}[1]{>{\centering\arraybackslash}p{#1}}
        \def\arraystretch{1.5}
        \begin{tabular}{P{67mm}p{14mm}P{67mm}}
            \cline{1-1}
            \cline{3-3}
            Benedikt Simbürger & & Vincent Sonvilla\\
        \end{tabular}

        \vspace{40mm}
        \begin{tabular}{P{67mm}p{14mm}P{67mm}}
            \cline{1-1}
            \cline{3-3}
            Nikolaj Voglauer & & Elena Widmann\\
        \end{tabular}
    \egroup

    \vspace*{\fill}
    \raggedleft Klagenfurt, am 04.04.2025

\end{center}

\newpage

\pagestyle{fancy}
\fancyhf{}
\fancyhf[EHL]{\includegraphics[width=0.2\textwidth]{HTL_Moessingerstraßen_Logo.png}}
\fancyhf[OHL]{\includegraphics[width=0.2\textwidth]{HTL_Moessingerstraßen_Logo.png}}

\fancyhf[EHC]{\begin{tabular}{cc}
    Simbürger & Sonvilla \\
    Voglauer & Widmann \\
\end{tabular}}

\fancyhf[OHC]{\begin{tabular}{cc}
    Simbürger & Sonvilla \\
    Voglauer & Widmann\\
\end{tabular}}

\fancyhf[EHR]{\color{mred} ELEKTROTECHNIK}
\fancyhf[OHR]{\color{mred} ELEKTROTECHNIK}

\fancyhf[EFL]{\todayD}
\fancyhf[OFL]{\todayD}

\fancyhf[EFC]{Automated-Factory-Storage-System}
\fancyhf[OFC]{Automated-Factory-Storage-System}

\fancyhf[EFR]{\thepage}
\fancyhf[OFR]{\thepage}


\newpage
\normalsize

\section*{Kurzbeschreibung}
\textcolor{blue}{
Beschreiben Sie das Projekt an dieser Stelle mit maximal zwei aussagekräftigen Sätzen, die dem Leser die Projektidee kompakt vermitteln und eine thematische Zuordnung ermögli-chen.
Die Kurzbeschreibung, im Umfang von einer A4-Seite, umfasst die wesentlichen Aspekte des Projektes in sozialer und technischer Hinsicht. Die Zielgruppe der Kurzbeschreibung sind auch Nicht-Techniker! 
Diese Beschreibung wird für Wettbewerbe und PR-Aktivitäten verwendet. Diese Inhalte sind unter dem Menüpunkt „About“ bzw. „Beschreibung“ auf der Projekthomepage dar-zustellen. Auf die Formulierung der Kurzbeschreibung sollte sehr viel Wert gelegt werden, weil viele Leser oft nur diese Seite lesen und den Rest lediglich durchblättern. Erstellen Sie die finale Kurzbeschreibung erst nach der Fertigstellung der Diplomarbeit.}

\color{blue}
\subsection*{Aufgabenstellung}
\begin{itemize}
    \item Warum ist die Themenstellung von Interesse?
    \item Was ist die vorgegebene Zielsetzung?
    \item Welche Ergebnisse sind zu erreichen?
\end{itemize}

\subsection*{Realisierung}
\begin{itemize}
    \item Von welchem Stand der Technik im Umfeld der Aufgabenstellung wurde ausgegan-gen?
    \item Welche Lösungsansätze sind grundsätzlich möglich?
    \item Warum wurde ein bestimmter Lösungsansatz gewählt?
    \item Welche experimentelle, konstruktive oder softwaretechnische Methodik wurde angewendet?
    \item Auf welche fachtheoretischen Grundlagen wurde aufgebaut?
    \item Welche wirtschaftlichen Überlegungen wurden angestellt?
\end{itemize}

\subsection*{Ergebnisse}
\begin{itemize}
    \item Worin besteht der konkrete Beitrag zur Lösung der Aufgabenstellung (Prototyp, Entwurfsplanung, Softwareprodukt, Businessplan etc.)?
    \item Kann das Ergebnis durch eine typische Grafik, ein Diagramm bzw. ein Foto illus-triert werden?
    \item Kann in die Vollversion der Diplomarbeit Einsicht genommen werden?
\end{itemize}

\color{black}
\vspace*{\fill}
\section*{}

\bgroup
    \def\arraystretch{1.5}
    \begin{tabular}{p{48mm}p{113mm}}
        \textbf{Kurztitel:} & \textcolor{blue}{Kurztitel der Diplomarbeit (etwa. 30 Zeichen)}\\
        \textbf{Schlüsselwörter:} & \textcolor{blue}{Maximal fünf aussagekräftige und durch Kommas getrennte Schlüsselwörter, welche die Inhalte möglichst gut beschreiben.}
    \end{tabular}
\egroup

\newpage
\pagenumbering{arabic}

\section*{Abstract}
\textcolor{blue}{Die Kurzbeschreibung in englischer Sprache sollte sinngemäß exakt der Kurzbeschreibung in deutscher Sprache entsprechen, siehe Abschnitt „Kurzbeschreibung“.}

\vspace*{\fill}
\section*{}

\bgroup
    \def\arraystretch{1.5}
    \begin{tabular}{p{48mm}p{113mm}}
        \textbf{Short title:} & \\
        \textbf{Keywords:} & 
    \end{tabular}
\egroup

\newpage
\tableofcontents
\newpage
\section{Einleitung}

\color{blue}
Mit der Einleitung beginnt der inhaltliche Teil der Arbeit. Es ist die Wahl der Themenstel-lung und das Interesse an dieser Themenstellung zu begründen.\\
Beschreiben Sie die Ausgangslage und die vom Auftraggeber stammende Problemstel-lung. Leiten Sie daraus die konkrete Aufgabenstellung und Zielsetzung des Gesamtpro-jekts ab. Geben Sie einen kurzen Überblick über das fachliche und wirtschaftliche Projekt-umfeld. Stellen Sie gegebenenfalls Ihre Kooperationspartner (Firma, Kontaktpersonen, Kontaktdaten, Zuständigkeit, etc.) dar. \\
Die Leserinnen und der Leser bekommen durch dieses Kapitel einen Gesamteindruck über die vorliegende Diplomarbeit. An dieser Stelle kann es hilfreich sein, das Zusammenwirken unterschiedlicher Komponenten des Gesamtsystems grafisch darzustellen, z.B. mittels eines Blockschaltbildes der Systemstruktur oder ähnlicher Darstellungsformen. In der Sys-temübersicht sollten bereits die Zuständigkeitsbereiche der einzelnen Projektmitglieder ausgewiesen sein und im Text auch oberflächlich beschrieben werden, siehe Abbildung~\ref{Systemstruktur}.\\

\begin{figure}[h]
    \includegraphics[width=1\textwidth]{Systemstruktur.png}
    \centering
    \caption{Systemstruktur}
    \label{Systemstruktur}
\end{figure}

Stellen Sie kurz die in der Arbeit behandelten Themen vor und verweisen Sie auf die ent-sprechenden Abschnitte in der Diplomarbeit.\\
\color{black}

\subsection{Vision}
\subsection{Projektziel}
\subsection{Kooperationspartner}

\newpage
\section{Grundlagen und Methoden}
\color{blue}
Dieses Kapitel ist der gemeinsame Hauptteil Ihrer Diplomarbeit und ist üblicherweise in mehrere Unterkapitel unterteilt.\\
Beschreiben Sie die Ist-Situation des Standes der Technik und darauf aufbauend erläutern Sie mögliche Lösungsansätze, die zur Umsetzung in Frage kommen. Begründen Sie die ge-wählten Methode und Lösungsansätze.\\
In diesem Kapitel werden auch die Grundlagen behandelt, die für das gesamte Projekt essentiell sind und nicht einem einzigen Funktionsblock alleine zugeordnet werden kann.\\
Es ist die Gesamtsicht, die zur Lösung Ihrer Aufgabenstellung führt, darzulegen. Notwen-dige Schnittstellen zwischen den einzelnen Modulen können anhand eines detaillierteren Gesamtblockschaltbildes beschrieben werden (hier nicht dargestellt). \\
An dieser Stelle ist die Produkt- und die daraus abgeleitete Projektstruktur in From eine Scrum-Projektplans darzulegen, zu beschreiben und zu argumentieren.\\
Der Produktstrukturplan (PdSP) enthält die wichtigsten Systemkomponenten (Hard-ware/Software) des Gesamtsystems, weshalb in den Blöcken Hauptwörter stehen, die die Komponenten bezeichnen, siehe Abbildung~\ref{Produktstruktur}. \\
Der Scrum-Projektplan beschreibt den geplanten Projektablauf, siehe Fehler! Verweisquel-le konnte nicht gefunden werden..  Die Vorlage für den Scrum-Plan und eine Beschreibung zum Aufbau befindet sich in der DA-PowerPoint-Vorlage. An dieser Stelle ist der Scrum-Plan nur bis zur Story-Ebene darzustellen, sodass die (grobe) Projektplanung beschrieben werden kann. Aus der Beschreibung soll die wichtigsten Arbeitspakete und die Verant-wortlichkeiten der Teammitglieder eindeutig hervorgehen. Die Epics (etwa 4 Wochen) und Stories (etwa 2 Wochen) sind kurz zu beschreiben. Die konkreten Tätigkeiten werden als Tasks bezeichnet und befinden sich in der dritten Ebene. Task-Beschreibungen müssen stets ein Verb beinhalten, z.B. „Platine bestücken und testen“. Der vollständigen Scrum-Projektplan inklusive der Tätigkeiten wird allerdings erst in Kapitel 8.1.2 eingefügt.\\
Um die Lesbarkeit der Darstellungen zu gewährleisten, kann es sinnvoll sein, diese im Querformat in Dokument einzufügen oder mehrere Teilbereiche in separaten Abbildun-gen darzustellen.\\

\begin{figure}[h]
    \includegraphics[width=1\textwidth]{Produktstruktur.png}
    \centering
    \caption{Produktstruktur}
    \label{Produktstruktur}
\end{figure}

Die Überschriften der nachfolgenden Unterkapitel stellen lediglich Platzhalter dar und können durch spezifische, aussagekräftigere Titel ersetzt werden (Formatierung als Über-schrift!).\\
2.1 Ist-Situation / Stand der Technik\\
2.2 Lösungsmethoden / Lösungsansatz\\
2.3 Grundlagen für das Gesamtprojekt\\
2.4 Schnittstellen zwischen den Modulen\\
2.5 Produktstrukturplan (PdSP)\\
2.6 Scrum-Projektplan (Epics und Stories)

\begin{figure}[h]
    \includegraphics[width=0.8\textwidth]{Scrum_Projektplan.png}
    \centering
    \caption{Scrum Projektplan}
\end{figure}

\color{black}
\newpage
\section{Stuff}

\color{blue}
Jedes Teammitglied bearbeitet die individuell vereinbarte Zielsetzung, im Kontext der Gesamtaufgabenstellung, eigenständig. Der individuelle Teil sollte pro Teammitglied ei-nen Seitenumfang von etwa 25 bis 30 Seiten umfassen.\\
Der Titel der individuellen Aufgabenstellung entspricht der Bezeichnung Ihrer „Individuel-len Themenstellung“, siehe Diplomarbeitsdatenbank (Genehmigung). Achtung, in der Kopfzeile der individuellen Kapitel wird ausschließlich der Name des jeweiligen Teammit-glieds (Urheber) angegeben!\\
Beschreiben Sie an dieser Stelle ausführlich die konkrete Aufgabenstellung zu Ihrer indivi-duellen Zielsetzung. Die Überschriften der nachfolgenden Unterkapitel stellen lediglich Platzhalter dar und können durch spezifische, aussagekräftigere Titel ersetzt werden. Die Arbeit sollte allerdings stets die drei Aspekte (Grundlagen/Methoden, Realisie-rung/Implementierung, Ergebnisse) in einer angemessenen Art und Weise beinhalten.\\

\begin{itemize}
    \item Eine reine Abschreibübung theoretischer Inhalte (copy and paste) wird nicht ange-strebt.
    \item Eine möglichst solide technische Darstellung der eigenen Arbeit hingegen ist äu-ßerst erwünscht.
    \item Grobstruktur der individuellen Lösung (inkl. Begründung): Blockschaltbild \(\rightarrow\) Schaltung  Flussdiagramme/Klassendiagramme \(\rightarrow\) Quellcode (auszugsweise) \textbar Anforde-rung \(\rightarrow\) Datenbankschema \textbar etc.
    \item Detaillierte Beschreibung der konkreten Umsetzung mit Referenzen auf relevanten Quellen (Primärliteratur/ Webseiten): Schaltungsentwicklung und Simulationen, Al-gorithmen, Messreihen und Auswertungen.
\end{itemize}

\section*{Gliederung der Arbeit und Inhalte}
Der individuelle Teil kann und soll nach eigenem Ermessen gestaltet werden. Ein Leitfaden für einen guten Stil und eine gute Dokumentstruktur finden Sie in Kapitel 7. Die Über-schriften sollten nicht zu allgemeingültig, sondern möglichst spezifisch und aussagekräftig für ihre Arbeit, gewählt werden. Inhaltlich sollten die folgenden Punkte in ihrer Arbeit enthalten sein:

\begin{itemize}
    \item Beschreiben Sie kurz, zum Beispiel den Stand der Technik, die Lösungsalternativen und begründen Sie die Wahl des Lösungsansatzes, die auf Ihre individuelle Zielset-zung zutrifft. 
    \item Beschreiben Sie alle theoretischen Betrachtungen und praktischen Tätigkeiten, die Sie während der Umsetzung Ihrer individuellen Zielsetzung durchgeführt haben. An dieser Stelle präsentieren Sie den eigentlichen Hauptteil Ihrer „Arbeit“.
\end{itemize}

\textbf{Theoretischer Teil:} Analysen von Markt/Systemen/Baugruppen/Sensoren/Bauteilen, Lösungswege, Berechnungen, Simulationen, Schaltungen, Funktionsbeschreibungen, Schnittstellen, Codeauszü-ge. \\
\textbf{Praktischer Teil:} Messaufgaben, Aufbauten, Messschaltungen, Fotos, Tabellen, Di-agramme, Tests, Ergebnisse, Vergleiche, Interpretationen\\
Besteht Ihre Arbeit aus mehreren Komponenten, ist es im Regelfall sinnvoll diese in je-weils eigenen Unterkapiteln zu behandeln. Die obigen Punkte würden sich in diesem Fall komponenten-spezifisch in den einzelnen Unterkapiteln „wiederholen“. \\
Jede Arbeit sollte mit einem Kapitel “Konklusion“ (maximal eine Seite) enden, in welcher die tatsächlich erzielten Ergebnisse kompakt zusammengefasst werden (Ergebnisse, Ver-gleiche, Interpretationen) und ein Ausblick für eine mögliche zukünftige Weiterentwick-lung des Projekts gegeben wird (Ausständiges, offene Probleme, Erweiterungen, etc.).\\
Die Überschriften der nachfolgenden Unterkapitel stellen lediglich Platzhalter dar und können durch spezifische, aussagekräftigere Titel ersetzt werden (Formatierung als Über-schrift!).\\
3.1 Aufgabenstellung\\
3.2 Grundlagen / Methoden\\
3.3 Realisierung / Implementierung\\
3.4 Ergebnisse / Konklusion\\

\color{black}
\newpage

\section{Hardwareentwicklung, Softwarebackend \\ und Benutzeroberfläche \textcolor{gray}{(Simbürger)}}

\subsection{Hardware}

\subsubsection{Planung}
Grundlegende Anforderungen zur planung der AFSS-Mechanik, sind die größe des Lifts an der \\HTL-Mössingerstraße, sowie möglichst einfache realisierung mit HTL-mitteln, möglichst wenig kompromisse in der Funktion oder zuverlässigkeit eingehen zu müssen.
Die Anforderung der Transportfähigkeit limitiert die Größe des Lagers auf 2.3 m Länge um im Lift transportiert werden zu können und auf 1.9 m höhe aufgrund der Türhöhe im Keller. Weiters müssen auch noch Rollen an den Rahmen angebracht werden, um das Lager überhaubt erst ohne großen mehraufwand bewegen zu können. Diese Extrahöhe limitiert den Rahmen weiter.
Nun soll dieser rund 2.25 m lange und 1.8 m hohen Raum optimal genutzt werden um eine möglichst große Lagerdichte sicherstellen zu können.
Um möglichst gute erweiterbarkeit, sowie eine Fertigung an der Schule zu ermöglichen, sollen für die mechanische Trägerkonstruktion sog. Item-Profile verwendert werden. 

\paragraph{Item}

\begin{wrapfigure}{r}{0.3\textwidth}
    \vspace{-5mm}
    \includegraphics[width=0.3\textwidth]{Item-Standartverbindungssatz.png}
    \caption{Item Profil mit Standartverbindungssatz \cite{Item_svs}}
\end{wrapfigure}

Das Item-Profilsystem, ist ein System welches Aluminium Extrusionen in verschiedenen Ausführungen, sowie viele Verbindungs-möglichkeiten zu sich selbst sowie andere mechanische Elemente bietet. Hierbei gibt es eine breite Auswahl an Profielen, von 20x20 mm bis 40x40 mm Querschnitt. Für alle Komponenten mit hoher mechanischer Beanspruchung werden 40x40-Extrusionen verbaut, da diese eine besonders hohe Biegefestigkeit aufweisen. Für Anwendungen mit geringerer Beanspruchung sowie aus Platz- und Gewichtsparmaßnahmen, werden 20x20-Extrusionen Verbaut. 
Zur Verbindung zu anderen Bauelementen gibt es die Möglichkeit sog. Nutenstene mit verschiedenen Gewinden in die Nut einzulegen und dort Platten o. ä. anzuschrauben. Um Item-Profile untereinander zu verbinden, werden Standartverbindungssätze verwendet. 
\\
Die Anwendung im AFSS erfordert weiters recht lange Verfahrwege. Um dies kostengünstig umsetzen zu können, werden V-Slot Profile verwendet


\paragraph{V-Slot}

\begin{wrapfigure}{r}{0.3\textwidth}
    \vspace{-5mm}
    \includegraphics[width=0.3\textwidth]{V-Wheel.png}
    \caption{V-Slot mit V-Wheel \cite{v_slot_wheel}}
\end{wrapfigure}

Auch V-Slot-Profile sind Aluminium Extrusionen. Diese können grundsätzlich auch ähnlich wie Item-Profile, mit Nutensteinen ect. verwendet werden, sind aber zusätzlich darauf ausgelegt, dass ein V-Wheel in einer Narbe des Profils rollen kann. 




\subsubsection{Rahmen}

Der Rahmen des AFSS bezeichnet jene Struktur, welche als äußerstes Gehäuse, sowie grundlegende mechanische Stabilität bietet. Es ist geplant, dass es die anforderungen von Maximalhöhe und -länge optimal ausnutzt sowie breite minimal hält.  

\subsubsection{Y-Achse}
\paragraph{Motorauslegung}

\subsubsection{X-Achse}
Die X-Achse des AFSS is 


\paragraph{Schlittenauslegung}



\paragraph{Antrieb}

\paragraph{Kabelführung}


\subsubsection{Z-Achse ("Gabel")}
\paragraph{Positionsbestimmung}

Um die optimale Postiton der Zahnriemenaufhängung für die Y-Achse bestimmen zu können, wird überschlagsmäßig ein Massenschwerpunkt in Z-Richtung berechnet. Um die Konstrukiton beginnen zu können, werden hierfür ungefähre werte angenommen.

\noindent\begin{minipage}{\textwidth}
\begin{minipage}[t]{0.5\textwidth}
    \vspace{7mm}
    \begin{equation*}
        X_s = \frac{1}{M} \cdot \displaystyle\sum_{i = 0}^{n} z_i \cdot m_i
    \end{equation*}
\end{minipage}%
\begin{minipage}[t]{0.5\textwidth}
    \begin{align*}
        &Z_s: \text{Z-Koordinate des Massenschwerpunkts} & &\left[m\right]\\
        &M: \text{Gesamtmasse} & &\left[kg\right]\\
        &z_1: \text{Z-Koordinate der Teilmasse} & &\left[m\right] \\
        &m_1: \text{Masse der Teilmasse} & &\left[kg\right]
    \end{align*}
\end{minipage}
\end{minipage} 
\vspace{5mm}

    \begin{table}[h!]
        \centering
        \centering
            \begin{tabular}{c c c}
                Gegenstand & Masse in kg & Position in mm\\
                \hline
                Motor & 0.3 & 21 \\
                Z-Schiene & 1 & 150 \\
                Z-Gable & 0.3 & 180 
            \end{tabular}
        \caption{X-Achse unbeladen und eingefahren}
        \end{table}
    \begin{table}[h!]
            \centering
            \begin{tabular}{c c c}
                Gegenstand & Masse in kg & Position in mm\\ 
                \hline
                Motor & 0.3 & 21 \\
                Z-Schiene & 1 & 150 \\
                Z-Gabel & 1.3 & 380
            \end{tabular}
            \caption{X-Achse beim Ladevorgang}
        \end{table}

        \vspace{5mm}
        So werden zwei Schwerpunkte berechnet: ca. 130 mm im unbeladen und 250 mm während dem Ladevorgang. Da die Stabilität des Y-Schuttels während dem Ladevorgang wichtiger ist als während einer Leerfahrt wird das Y-Shuttle so positioniert, dass die Aufhängung des Zahnriemen bei rund 200 mm liegt. 
    

\subsubsection{Fertigung}

\paragraph{Umlenkrollen}
Die Umlenkrollen sind jeweils am Ende der X- und Y-Achsen angebracht. Dadurch dass diese der gesamten Spannkraft ausgesetzt. Dies erfordert spezielle Anforderungen an die Aufhängen als auch an die Umlenkrolle selbst. Diese soll primär eine 180° Wende des Zahnriemens ermöglichen, sowie sekundär eine Führung für den Riemen bieten. 
Umgesetzt wird diese Anforderungen, durch Fertigung von vier Aluminium-Drehteilen in welche Kugellager eingepresst werden.

\begin{figure}[h]
    \centering
    \begin{subfigure}{.6\textwidth}
        \centering
        \includegraphics[width=10cm]{AT5x16-Umlenkung.png}
        \caption{Bauteilzeichnung Umlenkrolle}
        \label{UmlenkrolleBTZ}
        
    \end{subfigure}%
    \begin{subfigure}{.4\textwidth}
        \centering
        \includegraphics[width=5cm]{AT5x16-Umlenkung-Dorn.png}
        \caption{Dorn}
        \label{DornBTZ}
    \end{subfigure}
    %\caption{Dorn}
    %\label{fig:test}
    \end{figure}

Die Fertigung dieses Teils Lässt sich in folgende Teilschritte unterteilen:
\begin{itemize}
    \setlength\itemsep{0mm}
    \item Zuerst die Frontfläche Plandrehen (900 U/min)
    \item Ungefär 30mm Länge auf das Aussenmaß von 36mm Plandrehen
    \item Mit 9.8mm Bohrer das mittlere Loch vorbohren (540 U/min)
    \item Mit 10mm Reibeisen und viel Öl das Loch auf eine genaue passung bringen (260 U/min)
    \item Die Position Relativ zum Backenfutter makieren um beim neu einspannen Rundlaufgenauigkeit gewährzuleisten
    \item Zylinder bei ca. 26 mm Abstechen (540 U/min)
    \item Umspannen und auf maß plandrehen (900 U/min)
    \item Die Aussparung für die Lager mit Eckdrehmeissel beginnen, jedoch nach innen hin nur 6.8 mm 
    \item Bei ca. 17 mm Lochdurchmesser den tatsächleichen Durchmesser mit der Digitalanzeige vergleichen und gegebenennfalls korrigieren
    \item Bei 21.5 mm den Oberschlitten die restlichen 0.2 mm zustellen und die gesamte Tiefe plandrehen
    \item Den Lochdurchmesser auf 21.95 erweitern, und dann in kleinen inkrementen zustellen bis das Lager gerade so nicht passt, um einen Pressitz zu gewährleisten, dies tritt bei Lagern mit 22 mm Aussendurchmesser bei rund 22.04 mm ein.
    \item Da für die Einsparung der Riemenführungsfläche kein Angriffspunkt verfühbar ist wurde auls Halterung ein Dorn gedreht, auf welchen das Drehteil aufgeschraubt wird.
    \item Mit dem Abstechmeisel wird in 2 mm inkrementen die Zahnriemenauflagefläche herausdrehen, bis auf 30.2 mm, sowie links und rechts den Rand 1mm extra dick lassen. (540 U/min)
    \item Am Schluss wird der Rand auf Maß gedreht und die Tiefe fertiggedreht. 
    \item Als letzten Schritt werden links und rechts die zwei lager eingepresst.
\end{itemize}

    Durch die Verhältnissmäßig großen Toleranzen bei den Lageraussendurchmessern wird bei 2 der 8 Lagerpassungen zusätztlich Lagerkleber verwendet um eine Zuverlässige Passung zu gewährleisten.


\paragraph{Lasern}
\paragraph{Fräsen}

\subsubsection{Aufbau}

\subsection{Software und Benutzeroberfläche}

\subsubsection{Grundlegendes}

\subsubsection{Aufbau}

\subsubsection{Benutzeroberfläche}



\subsubsection{Datenbanken}

Als Datenbanksystem wird aufgrund des guten Supports MySQL gewählt. Dies ist ein relationales Datenbankmanagementsystem welches in einem Docker-Container aufgesetzt wird. In diesem werden alle Daten gespeichert, die zur Auswahl sowie zur Ausliferung von Teilen nötig sind.

\begin{figure}[h]
    \includegraphics[width=0.6\textwidth]{DB-Schema.png}
    \centering
    \caption{Datenbankschema des AFSS}
\end{figure}

Wie in Abb. 4.2 ersichtlich, beinhäklt sich diese Datenbank fünf Tabellen. Diese hohe Komplexität resultiert daraus, dass diese Struktur eine 100\%ige Flexivilität in der Ablage von Bauteilen in einem überliegendem System bietet.

Die Erste Tabelle beschreibt ein einziges theoretisches Bauteil. Dieses hat einen Namen, Gewicht, Beschreibung und Kategorien zur Filterung. Unter der Spalte 'picture' wird ein Dateiname gespeichert, der zu einem Bild zeigt, dass das Produkt abbildet.
Die zweite beschreibt einen Container. Im Lagersystem entspricht dieser einer Box. Diese kann mehrere `stocks' beinhalten, sowie durch einen Barcode identifiziert werden. Weiters muss jeder Container immer eine aktuelle Position (`current\_location') besitzen, an der die Box gerade ist. Im ausgelagerten (und noch nicht eingelagerten) Zustand ist diese `location' Position 0. Das Ziel der Box wird in `target\_location' gespeichert. Stimmt die aktuelle mit der Zielposition überein, so ist die Box an ihrem Ziel angelangt. Die Kategorie `size' beschreibt die Größe eines Containers und lässt somit theoretisch zu, dass in Zukunft auch unterschiedlich große Boxen zuverlässig in die richtigen Lagerplätze eingelagert werden. `priority' wird nicht verwendet.

Container und Artikel werden im sog. 'stock' verheiratet. Dieser kann als bauteilhaufen in einer Box verstanden werden. Es können also auch mehrere 'stocks' mit dem selben Container geben, dies würde mehreren verschiedenen Bauteilen in einem einzigen Copntainer entsprechen. Auch ist es möglich mehrere Container mit den selben 'stocks' abzubilden, welches ener aufteilung von Abuteilen auf mehrere Container entspräche.

Die Positionen der Container werden in Standorte ('locations') abgebildet. Diese entsprechen den Lagerplätzen. Sie sind einer darüberliegenden 'area' zugeordnet, welche einerseits einen Lagerschrank, aber weiters auch Module wie Vereinzelungsanalgen, abbilden kann. Standorte verfüghen weiters über eine Position welche in X, Y und Z-Richtung beschreibt, wo sich ein Standort im Referenzsystem des Lagers befindet. Auch die größe des Lagerplatzes wird abgebildet, um sicherzustellen, dass auf jeden fall nur die richtige Größe an Box eingelagert wird.


\subsubsection{Docker}

    \begin{wrapfigure}{r}{0.3\textwidth} % 'r' for right, 'l' for left
        \vspace{-20px}
        \includegraphics[width=0.3\textwidth]{docker-logo-011.png} % Replace with your image file
        \caption{Docker-Logo: \cite{docker_logo}}
    \end{wrapfigure}

    Docker ist eine Umgebung, in der Softwareprojekte isoliert werden können. Da besonders auch bei Projekten mit großem Python anteil, viele Pakete mit verschiedenen Versoinen benötigt werden, ist es sehr hilfreich diese zu bündeln. \\
    Umgesetzt wird dies mithilfe von Containern welche einen gesamten Programmteil als alleinstehende Einheit enthält. Diese werden über ein 'Dockerfile' configuriert, welches sich im selben Ordner wie die Python-Anwendung befindet. In diesem werden Parameter wie die Python-Version und die benötigten pip-Pakete sowie den Programmeinstiegspunkt angegeben. \\ 
    Ein zweier Docker Conteianer wird mit einem MySql-Image erstellt, dort wird die Datenbank aufgesetzt. \\
    Um diese Zwei Container miteinander Kommunizieren zu lasen, ist es ntig ein sog. docker-compose.yml File zu erstellen. Dies enthält alle Informationen über verwendete container, deren Ports, sowie Speicher für Dateien (Volumes). Bei Testbetrieb wird der Datenbankcontainer aleinstehend bestrieben und mit einem anderen Port, keine Zugriffsprobleme zu generieren. In Produktion wird dann derselbe Container in den Containerverband übertragen und dort mit einem anderen Port weiterverwendet. \\
    Erstellt wird dieser Containerverband mit den Consolenbefehlt der auf das docker-compose File zugreift. 
    \begin{lstlisting}[language=bash]
        docker build docker-compose.yml\end{lstlisting}
    Dann werden auch alle Logs in der Kommandozeile ausgegeben sowie 


\subsubsection{Backend}
\subsubsection{APIs}
\subsubsection{Lageralgorithmus}
\subsubsection{Artikelsuche}
\paragraph{TF-IDF und Rust Implementierung} \hspace{0pt} \\
Der TF-IDF (Term Frequency-Inverse Document Frequency) Algorithmus, ist ein Weg um wichtige wörter aus Dokumenten zu extrahieren. Er wird verwendet um beispielsweise in Suchmaschienen, eine Suchanfrage mit Webpagecontent abzugleichen, und die am besten mit der Suchanfrage übereinsteimmenden Dokumente zu sortieren. \\
Im Fall dieser Anwendung werden die Daten aus der Artikeldatenbank als 'Dokumente'  angesehen und die Suchanfrage aus dem Suchfeld wird dafür verwendet um die am besten passenden Artikel zu finden.
\\
Durch den Relativ hohen Rechenaufwand bei dieser Suchoperation wird dieser in der Programmiersprache Rust implementert. Die Implementierung in Rust ist im vergleich zu Python schon bei relativ kleinen Datenmenge bis zu 5-mal schneller.

\subparagraph{Rust}
Rust ist eine sehr effiziente und schnelle Programmiersprache die in den späten 2000er und frühen 2010ern bei Mozilla und der Open-Source-Community entwickelt. Sie unterstützt unter anderem mehr Typensicherheit und verhindert viele Programmierfehler. 

Die Funktion dieses Algorithmus ist in drei unterteile Unterteilt.

\begin{enumerate}
    \item Term Frequenz \\
    Die Termfrequenz gibt an, wie oft ein angegebenes Wort in einem Dokument vorhanen ist. Dies wird durch die Folgende Funktion kalkuliert.
    
    \begin{lstlisting}[language=Rust]
fn term_frequency(document: &str, term: &str) -> f64 {
    // Store the lowercase document as a String to ensure it lives long enough
    let lower_document = document.to_lowercase(); 

    // Split the document into words
    let normalize_document: Vec<&str> = lower_document.split_whitespace().collect();
    // Make sure the searchterm is lowercase
    let normalize_term = term.to_lowercase();

    // Count occurrences of the term in the document
    let count = normalize_document
        .iter()
        .filter(|&&word| word == normalize_term) // Compare each word with the term
        .count();

    // Calculate the term frequency as occurrences / total number of words
    let total_words = normalize_document.len();
    if total_words == 0 {
        0.0 // Avoid division by zero if the document is empty
    } else {
        count as f64 / total_words as f64
    }
}\end{lstlisting}

    Mithilfe dieser wird eine Liste aller Wörter und der Vorkommenshäufigkeit dieser erstellt.

    \item Die zweite Komponente ist dann die Inverse Dokument Frequenz. Diese gewichtet, die Anzahl der Dokumente in dem das gesuchte Wort enthalten ist relativ zur Gesamtdokumentanzahl vorkommt. Häufig vorkommende Worte wie z.B. 'und' werden hierbei weniger gewichtet als einzigartige Wörter.

\begin{lstlisting}[language=Rust]
fn inverse_document_frequency(term: &str, all_documents: &Vec<String>) -> f64 {
    let mut num_documents_with_this_term = 0;

    // Iterate over all documents to check if they contain the term
    for doc in all_documents {
        // Normalize both term and document by converting them to lowercase
        let lower_doc = doc.to_lowercase();
        let normalized_doc: Vec<&str> = lower_doc.split_whitespace().collect();

        // Check if the term exists in the document
        if normalized_doc.contains(&term.to_lowercase().as_str()) {
            num_documents_with_this_term += 1;
        }
    }

    // Calculate IDF
    if num_documents_with_this_term > 0 {
        // Apply the IDF formula: 1 + log(total_documents / documents_with_term)
        1.0 + ((all_documents.len() as f64) / (num_documents_with_this_term as f64)).ln()
    } else {
        // If the term is not found in any document, return 1.0
        1.0
    }
}
\end{lstlisting}

    \item Nun liegt Liste davon vor, wie oft ein Wort in den Suchdaten vorkommt, als auch, wie oft ein Suchbegriff in einem bestimmten Dokument ist. \\ Als nächsten Schritt werden diese beiden Werte für jeden Suchbegriff miteinander multipliziert und ergeben somit einen Vektor der die Suchwörter in Relation zu jedem einzelnen Dokument stellt.
    \item Als letzten Schritt wird der zuvor errechnete Dokumentenvektor (der IDF jedes Suchterms in jedem Dokument) mit dem Suchvektor verglichen. Die geschieht mit der sog. Kosinus-Ähnlichkeit.
    \begin{lstlisting}[language=Rust]
fn cos_similarity(query_p: Vec<f64>, document_p: Vec<f64>) -> f64 {
    // Ensure that both vectors have the same length
    if query_p.len() != document_p.len() {
        return -1.0;
    }

    let mut dot_product = 0.0;
    let mut abs_doc_squared = 0.0;
    let mut abs_query_squared = 0.0;

    // Calculate the dot product and the magnitudes (squared)
    for i in 0..query_p.len() {
        dot_product += query_p[i] * document_p[i];
        abs_doc_squared += document_p[i].powi(2); // document_p[x] ** 2
        abs_query_squared += query_p[i].powi(2); // query_p[x] ** 2
    }

    // Calculate the magnitudes
    let abs_doc = abs_doc_squared.sqrt();
    let abs_query = abs_query_squared.sqrt();

    // Handle division by zero in case of zero vectors
    if abs_doc == 0.0 || abs_query == 0.0 {
        return 0.0;
    }

    // Return the cosine similarity
    return dot_product / (abs_doc * abs_query);
}\end{lstlisting}
    Nach der Berechnung dieser für jedes Dokument werden alle Dokumente sortiert und ausgegeben.
\end{enumerate}








\newpage
\section{Titel der Individuellen Themenstellung (Vincent Sonvilla)}
war hier
\newpage
\section{Titel der Individuellen Themenstellung (Nikolaj Voglauer)}

\newpage
\section{Titel der Individuellen Themenstellung (Elena Widmann)}

\newpage
\section{Anhang}
\textcolor{blue}{Im Anhang befinden sich weitere Detailinformationen des Projekts wie\\
•	Datenblattauszüge, Fertigungsunterlagen (PCB-Layouts, Gehäusezeichnungen, 3D-Druckunterlagen, Montageanleitungen,…) etc.\\
•	sämtliche geforderten Projektmanagementdokumente\\
•	ein Businessplan (optional)
}

\subsection{Projektmanagement}
\textcolor{blue}{In diesem Kapitel soll auf das Projektmanagement des Projektes eingegangen werden. Zu Beginn empfiehlt es sich, die einzelnen Bereiche des Projektmanagements zu erklären und anschließend in einzelnen Kapiteln zu behandeln.}

\subsubsection{Aufgabenstellung des Gesamtprojekts}
\textcolor{blue}{Fügen Sie an dieser Stelle den Text der genehmigten Aufgabenstellung ein, der in die Diplomarbeitsdatenbank  eingegeben wurde.}

\subsubsection{Scrum-Projektplan}
\textcolor{blue}{Fügen Sie hier den vollständigen Scrum-Projektplan, wobei die Nummern der Tasks mit der Arbeitszeitaufzeichnung übereinstimmen müssen. Der Scrum-Projektplan kann auf mehrere Seiten aufgeteilt werden.}

\bgroup
    \centering
    \includegraphics[width=0.6\textwidth]{Scrum_Projektplan_mit_Tasks.png}
    \captionof{figure}{Scrum Projektplan mit Tasks}
\egroup

\newpage
\subsubsection{Terminplanung}

\newpage
\subsection{Inbetriebnahme}
\color{blue}
Nachdem typische Projekte aus mehreren Komponenten bestehen, ist es oft nicht trivial die einzelnen Komponenten korrekt zu konfigurieren und das Gesamtsystem in Betrieb zu nehmen. In diesem Kapitel soll eine vollständige, präzise und trotzdem möglichst kompak-te Anleitung zur Inbetriebnahme des Systems dargelegt werden. Die Schritte sollen in dem Detailgrad beschrieben werden, dass ein durchschnittlicher Schüler des vierten Jahrganges das Projekt in Betrieb nehmen kann. Exemplarisch sollten Punkte wie die folgenden be-handelt werden – die Aufzählung ist nicht vollständig):
\begin{itemize}
    \item Treiberinstallationen und Systemkonfigurationen
    \item Zu empfehlen wäre bei Server-Installationen ein Setup-Script, welches auf einem vordefinierten Docker-container aufbaut.
    \item Welche Schritte sind notwendig, um das Projekt mit dem vorhandenen Code / Schaltplänen (auf GIT, CD, Netzlaufwerk, etc.) in Betrieb zu nehmen.
    \item Bei Schaltungen mit mehreren Platinen muss beschrieben werden, wie diese mit-einander verbunden werden müssen.
\end{itemize}
\color{black}

\newpage
\subsection{Kostenaufstellung}
\textcolor{blue}{Für die Kalkulation im Gesamtprojekt sind folgende Kosten zu erfassen: \\
•	Kosten für Material (Hard- und Software)\\
•	externe Kosten (z.B.: Zukauf von Sensoren, Funkmodule, spezielle Entwicklungsum-gebungen, etc.) 
}
\begin{figure}[h]
    \includegraphics[width=0.8\textwidth]{Kostenaufstellung.png}
    \centering
    \caption{Kostenaufstellung}
\end{figure}

\newpage
\subsection{Besprechungsprotokolle}
\textcolor{blue}{Eine entsprechende Vorlage wird auf den Schulrechner in Form einer Word-Vorlage be-reitgestellt. Scannen Sie die vier Besprechungsprotokolle (laut Vorlage) ein und fügen Sie diese nachfolgend an (eine A4-Seite pro Protokoll)\\
Protokolle zu den einzelnen Iterationen:\\
•	Vorprojektphase\\
•	Iteration 1\\
•	Iteration 3\\
•	Iteration 5\\
Die Termine und Inhalte/Projektstatus entsprechen denen der Iterationspräsentationen.
}

\newpage
\subsection{Arbeitsnachweis}
\textcolor{blue}{Jedes Teammitglied (Schüler/in) hat einen vollständigen Arbeitszeitnachweis, der außer-halb des Unterrichts verrichteten Tätigkeiten, in tabellarischer Form zu erbringen. \\Eine entsprechende Vorlage wird auf den Schulrechner in Form einer Excel-Vorlage bereitgestellt.}

\newpage
\subsection{Wettbewerbe}
\textcolor{blue}{Teilnahme und Erfolge bei Wettbewerben.}

\newpage
\subsection{Businessplan (optional)}
\textcolor{blue}{Halten Sie sich, was die Struktur des Businessplanes angeht, an die Angaben, die Sie im WIRE-Unterricht bekommen. Das Anfügen eines Businessplans ist durch den/die WIRE Lehrer/in vorab zu genehmigen.}

\newpage
\addcontentsline{toc}{section}{Literaturverzeichnis}
\printbibliography[title=Literaturverzeichnis]

\newpage
\renewcommand{\cftfigpresnum}{Abb. }
\setlength{\cftfignumwidth}{2cm}
\listoffigures

\newpage

\renewcommand{\cfttabpresnum}{Tab. }
\setlength{\cfttabnumwidth}{2cm}
\listoftables

\end{document}