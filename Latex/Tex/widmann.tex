\section{Sensorik und Sicherheitstechnik \textcolor{gray}{(Elena Widmann)}}

\subsection{Aufgabenstellung}

\subsection{Sensorik}

\subsubsection{Endschalter}
Beim Verplanen der Endschalter ist zwischen Software- und Hardware-Endschalter zu unterscheiden. Die Software-Endschalter begrenzen den Arbeitsbereich der Achse und sollten innerhalb des Bereichs der Hardware-Endschalter parametriert werden. Ihre Positionen werden direkt im Siemens TIA-Portal eingestellt und können falls notwendig einfach auf die aktuelle Geschwindigkeit angepasst werden. Werden die Software-Endschalter angefahren wird der Technologiealarm 533 ausgelöst und die Dynamikwerte werden gestoppt, das Technologieobjekt bleibt hierbei freigegeben. Werden sie jedoch überfahren wird das Technologieobjekt gesperrt. \\
Die Hardware-Endschalter begrenzen den maximal zulässigen Verfahrensbereich der Achse. Bei ihnen wird nicht unterschieden, ob die Endschalter angefahren oder überfahren werden. Beim Anfahren der Schalter wird der Technologiealarm 531 ausgelöst. Er sperrt das Technologieobjekt und muss, bevor der Auslösebereich der Hardware-Endschalter wieder verlassen werden kann, quittiert werden. \cite{axis_manual}\\
Auf jeder der drei Achsen vom AFSS und auf dem Querförderer müssen Hardware-Endschalter montiert werden. Die Auswahl begrenzte sich hierbei auf die uns zur Verfügung gestellten Sensoren, welche ihren Funktion entsprechend auf den verschiedenen Positionen eingebaut werden.

\paragraph{Positionsschalter mit Rollhebel}
An der x-Achse werden als Hardware-Endschalter Positionsschalter mit Rollhebel verwendet. Davon besitzen drei jeweils einem Öffner- und einen Schließerkontakt \cite{schmersal_3}, wohingegen einer der Endschalter aus zwei Öffnerkontakten besteht \cite{schmersal_1}. Um Einheitlich zu bleiben und da es sicherheitstechnisch auch von Vorteil ist (Drahtbruchsicher) verwenden wir jeweils einen der Öffnerkontakte der Endschalter.

\paragraph{Induktive Endschalter}
Als Hardware-Endschalter an der y-Achse werden induktive Sensoren verwendet. Sie besitzen jeweils einen Öffner- und einen Schließerkontakt \cite{induktiv_sensor}, wir verwenden jedoch ersteres um Drahtbruchsicher zu sein.

\paragraph{Endtaster}

\subsubsection{Referenztaster}
\paragraph{Autodesk Fusion 360}

\subsubsection{Lichttaster}

\subsubsection{Barcode-Scanner}

\subsection{AS-Interface}

\subsubsection{Allgemeines}

\subsubsection{Programmierung im TIA-Portal}

\subsubsection{Unterverteilerplatine}


\subsection{Sicherheitstechnik}
\subsubsection{Grundanforderungen und Planung}
\subsubsection{Realisierung}