\section{Allgemeiner Teil}

\subsection{Ausgangssituation 1-2}

\subsubsection{Anforderungen}

Um die Automatisierung dieses Prozesses zu ermöglichen, soll ein System entwickelt sowie gebaut werden, welches eine lagernde Box automatisch zu einer Kommissionierstation bringt. An dieser soll die Möglichkeit bestehen, lagerndes Inventar anzufordern, Lagerbestand auszufassen oder aufzufüllen sowie Boxen wieder einzulagern.\\
Weiters muss dieses System erweiterbar sein, um zukünftig neuen Lagerplatz hinzuzufügen, als auch die Möglichkeit zu bieten, andere Systeme wie Bauteilvereinzelung einzubinden.\\
Auch soll das AFSS mobil sein – dies gilt sowohl für Mechanik als auch Elektrik, da es nicht am endgültigen Standort errichtet wird und um bei möglichen Umbauarbeiten einen einfachen Transport zu ermöglichen.\\
Die Rahmenbedingungen zur Umsetzung dieser Ziele sind stark davon geprägt, dass alle mechanischen Bauteile, die eigens gefertigt werden müssen, so geplant werden, dass dies mit HTL-Mitteln möglich ist.\\
Es soll nicht nur produzierbar, sondern auch reproduzierbar sein. Daher soll die Dokumentation der Funktion sowie des Umsetzungsprozess so erfolgen, dass die Weiterführung dieses Projekts durch andere Schülerinnen und Schüler möglich ist.\\



\subsection{Potentielle Lösungen 2}
\subsubsection{Lagermethoden 3/4}
Die Industrie gibt viele Vorbilder in, wie ein Boxenbasierendes Lagersystem aufgebaut sein kann. \\
Eine sehr Platzeffiziente Variante ist beispielsweise die PickEngine von KNAPP \cite{pickengine}. bei dieser Variante werden Boxen in mehreren Ebenen übereinander Gelagert. Auf jeder Ebene gibt es bewegliche Roboter, welche die Boxen abholen und zu einem Lift bringen, von dem aus die Box dann zur Kommisionierstation gelangt. Dieses System ist ist sehr Platzeffizient und ausfallsicher. Jedoch ist es schwer möglich ein solches System in Miniatur zu Bauen, da die Hardwarefertigung sehr komplex ist.\\
Eine weitere Möglichkeit wäre ein Rotierendes Magazin, in welchem die Boxen auf einem horizontalen Karusell gelagert sind. Wenn ein bestimmtes Produkt benötigt wird, werden die Platformen soweit weiterrotiert, bis auf die gewüschte Box zugegriffen werden kann. Das Prinzip dieses Systems ist zwar recht simpel, jedoch ist es es schwierig die Mechanik der Rotation mit HTL-mitteln so zu bauen, dass ein Dauerbetrieb möglich ist. \\



\subsubsection{Steuerung 4/9}

\subsubsection{Schaltschrank 4/9}


\subsection{Verfolgter Lösungsansatz 3/4}


\subsection{Methodik 5}

\subsubsection{Software 2}
\subsubsection{Fusion360}
Zur mechanischen Planung des AFSS wird vorrangig Fusion360 genutzt. Dies ist ein 3D-CAD (Computer-Aided-Design) Programm, welches eine breite Palette von Funktionen bietet, jedoch noch sehr bedienerfreundlich ist. Weiters ist es sehr nützlich, dass Fusion360 eine integrierte Cloud-Speicherung anbietet. So können die Konstruktionen von anderen Personen einfach eingesehen werden. 

\subsubsection{Hardware 2}

\subsubsection{Fertigung 1}

\subsection{Sicherheitstechnik 1}

\subsection{Kooperationspartner}