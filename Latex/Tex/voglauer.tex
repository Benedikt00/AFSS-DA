\section{Elektroplanung und Realisierung \textcolor{gray}{(Nikolaj Voglauer)}}

\subsection{Elektroplanung}
\label{sec:Elektroplanung}

\subsubsection{Einleitung - Grundanforderungen}
    Die grundsätzliche Zielsetzung bei der elektrischen Planung, war die Anforderungen so zu erfüllen, dass die Lösung einerseits die Anforderungen von Erweiterbarkeit und Mobilität erfüllen und andererseits in der Schule beziehungsweise in der Werkstätte produzierbar waren. Die Elektrik des AFSS befindet sich in einem umgebauten Serverschrank, dessen physische Limitierungen bei der Planung ebenfalls zu berücksichtigen waren. Darunter fällt beispielsweise, dass die Module in die Breite von den, nur in die Tiefe verstellbaren, Profilschienen begrenzt werden.\\
    In der Anlage sollten während dem Normalbetrieb alle Komponenten vor elektrischen Störungen geschützt sein. Der Fokus liegt hierbei auf dem Schutz von Messleitungen und Steuerleitungen, denn diese überliefern präzise Daten die nicht verzerrt werden sollen.\\ 
    In der Planung wurde stehts bedacht, dass die elektrischen Komponenten so verbaut werden, dass im Falle eines Fehlers sowohl Personen gut geschützt sind und betroffene Geräte leicht auszuwechseln sind.\\

\subsubsection{Elektrik spezififsche Anforderungen}
\label{sec:Elektrik spezififsche Anforderungen}

    \paragraph{Versogung}\mbox{}\\
    Zur Verfügung steht dem AFSS eine 3-phasige Wechselspannung mit 400V Außenleiterspannung. Damit direkt angesteuert werden kann nur der Asynchronmotor für das Fließband. Alle anderen Elemente brauchen eine andere Spannungsebene. Die, in Summe, sieben Schrittmotoren brauchen 24 V mit einem möglichen Dauersummenstrom von über 20A. Die Logik bestehend aus Siemens-SPS, mit Ein und Ausgangskarten sowie PTO-Karten, und einer ET200 mit Asi-Master. Die Logik benötigen ebenfalls 24 V und sollen getrennt versorgt werden, um von potentiellen Fehlern bei den Schrittmotoren geschützt zu sein. Der Asi-Kreis benötigt eine eigene Asi-24V-Versorgung.

    \paragraph{Ansteuerungen}\mbox{}\\
    Angesteuert werden müssen 8 Motoren: 1 Asynchronmotor (250 W), 4 stärkere Schrittmotoren (2 Nm) und 3 schwächeren Schrittmotoren (40 Ncm). \\
    Der Asynchromotor soll keine Drehzahlregelung haben und über eine Wendeschützschaltung angesteuert werden. Die Schrittmotoren sollen über Schrittmotortreiber angesteuert werden. Diese Treiber werden von den PTO-Karten der SPS angesteuert.

    \paragraph{Sicherheit}\mbox{}\\
    Für die Anlage soll ein Fehlerstromschutzschalter (FI), ein Leitungsschutzschalter (LS), ein Motorschutzschalter und für jeden Schrittmotor eine Gleichstromsicherung ausgelegt werden.\\ 
    Um die Anlage trotz Fehler, die potentiell von den elektrischen Schutzeinheiten nicht erkannt werden, nach wie vor Abschalten zu können soll die Anlage über mehrere Not-Aus-Schalter verfügen. Zwei auf der Anlage selsbt, einen im Serverschrank/Schaltschrank und einen am Kommisionierplatz. Diese Positionierung soll es NutzerInnen ermöglichen aus jeder Position an der Anlage, einen Not-Aus-Schalter zu erreichen.

    \paragraph{Bedienelemente}\mbox{}\\
    Physische Bedienelemente wären beim AFSS ein Schlüsselschalter, zur Freigabe, und ein dreiphasiger Drehstromschalter, für eine manuelle Freischaltungsoption.

    \paragraph{Schaltschrank}\mbox{}\\
    Grundsätzlich haben Schaltschränke genormte Anforderungen (IEC 60208 und IEC 61439).\\
    Dazu gehört eine Auslegung von Kabelkanäle, die die Kabel schützen soll und Umbauten nicht zusätzlich erschweren sollen. Freifliegende Kabel sollen unter allen Umständen verhindert werden. Das Gehäuse muss geerdet sein und die inneren Komponenten vor Staub und Schmutz schützen. Bei einem potenziellen Lichtbogen soll der Schaltschrank Personen in der Nähe schützen. Zudem muss der Schrank gegen thermische Einflüsse geschützt sein, gegebenenfalls soll der Schaltschrank über eine Belüftung verfügen.\\
    Der Serverschrank, für den wir uns entschieden haben, schütz gegen Staub und Schutz und kommt mit einer Lüfteranlage, die die Abwärme von mehereren Gleichrichtern gut abführen kann. Zudem sind die Materialen des Schrankes vor Korrosion geschützt. \cite{Schaltschrank-Anforderungen} \\
    Bei der Planung muss beachtet werden, dass die Erdung aller leitungsfähigen Elemente eingehalten wird. Außerdem dürfen Umbauten wie die Montage von Rädern keine der angeführten Anforderungen widersprechen.

    \paragraph{Kabelauslegung}\mbox{}\\
    Beim Auslegen von Kabeln gibt es mehrere Punkte, die beachtet werden müssen. Während Spannungsabfall bei den Längen des AFSS vernachlässigt, werden können muss besonders auf Schleppkettentauglichkeit geachtet werden. Steuer- und Messkabel müssen entsprechend geschirmt werden und abhängig vom Strom muss der Querschnitt gewählt werden. Dabei gehören die Querschnitte aber auch auf die Schutzautomaten im Stromkreis abgestimmt.\\

    \paragraph{Module}\mbox{}\\
    Die Paneele/Module, auf welchen die elektrischen Komponenten montiert werden, müssen ebenfalls alle Erdungserwartungen erfüllen und mechanisch den Belastungen standhalten. Dabei ist das Gewicht die gravierenste Belastung. Eine gerechte Verdrahtung muss gewährleistet sein und die Modularität der Paneele soll vorteilhaft ausgenutzt werden und sollen nicht das Projekt unnötig verkomplizieren. Kostentechnisch soll dabei ein möglichst billiges, aber standhaftes Material gewählt werden.

\subsubsection{Mechanische Planung}

    \paragraph{Schaltschrankrahmen}\mbox{}\\
    \begin{wrapfigure}{r}{0.5\textwidth}
        \vspace{-30px}
        \includegraphics[width=0.5\textwidth]{Vogis Bilder/Serverschrank_original.jpg}
        \caption{Serverschrank bei Übergabe an das AFSS-Team}
        \vspace{-10px}
        \label{fig:Serverschrank_original}
    \end{wrapfigure}
    Bevor man sich den Details des Schaltschrankes widmet, muss vorerst der Rahmen festgelegt werden. Grundsätzlich würde man einen herkömmlichen Schaltschrank verwenden, allerdings sind diese teuer und erfüllen auch nicht unsere Anforderung der Mobilität. Deswegen wurden mehrere alternative Optionen in Betracht gezogen. Anfangs wurde der Ansatz verfolgt die elektrischen Komponenten in den Lagerschrank selbst einzubauen. Dabei hätte man entweder einen eigenen Abteil ins AFSS einplanen können, der auch mittels entsprechendem Material räumlich getrennt wäre, oder man hätte die Elemente fluide unterbringen können. Mit Letzteren ist gemeint, dass in dem ganzen Schrank verteilt die elektirschen Komponenten montiert wären.\\
    Die fluide Variante benötigt sorgfälltige Planung und auch sorgfälltige Absprache mit der restlichen mechanischen Planung des AFSS. Dafür hätte man eine kompacktes Design aber das Risiko für Verletzungen und Schäden wäre höher, da man die elektrischen Komponenten nur schwer räumlich trennen könnte, so wie es die Wände eines Schaltschrankes machen. Aufbauend auf dem fehlenden Sicherheitsaspekt und der Tatsache, dass die benötigte Kommunikation in einer Entwicklungsphase nicht möglich wäre, wurde sich gegen die fluide Option entschieden.\\
    Wesendlich realistischer ist der Ansatz ein eigenes kleines Abteil in den Schrank einzubauen. Der Rahmen wäre, sowie das Lager, aus Aluminiumprofilen gebaut und die jeweiligen Seiten würden mit Kunstsoffen verkleidet werden. Um die räumliche Trennung zu gewährleisten. Diese Option wäre Kommunikationstechnisch möglich, da man sich mit der restlichen mechanischen Planung nur auf die Außenmase und Position dieses Abteils einigen müsste. Allerdings bedeutet ein intigriertes Abteil auch weniger Lagerplätze und einen merkbaren Zusatzaufwand in der Realisierung. Da so viele Lagerplätze wie möglich verwirklicht werden sollen wurde sich gegen diese Variante entschieden.\\
    Auf der weiteren Suche nach einer Alternative zum herkömmlichen Schaltschrank wurde die Möglichkeit, einen alten Serverschrank zu recyclen erkannt. Ein fertig gebauter Schrank, der im Fehlerfall die Umgebung ausreichend schützt und keine Zusatzkosten mit sich bring war die beste Option. Die im Inneneren bereits vorhandenen Profilschienen bieten viele Möglichkeiten für die Elektrik montiert zu weden. Zudem bietet der Innenraum des Serverschrankes viel Platz und auch die Möglichkeit die Elektrik, wenn nötig, zügig zu erweitern, da man die Profilschienen sehr indiviuell nutzen kann.\\
    Es wurde sich somit entschieden den Serverschrank zu einem Schaltschrank umzubauen. Im Zuge des Umbaus würde der Innenraum umgebaut werden und der Serverschrank müsste auch mobil gemacht werden. 
    \paragraph{Modulprinzip}\mbox{}\\
    Im Zuge dieses Projektes wurde das Innenleben des Schaltschrankes auf mehrere Module getrennt. Die Anforderungen an diese wurden schon beschrieben, doch ursprünglich waren weitere Alternativen für den Innenraum des Serverschrankes in Diskussion.\\
    Anstatt von mehreren Modulen, die später genauer beschrieben werden, könnte man eine durchgehende Platte verwenden und diese an die Profilschienen des Serverschrankes festschrauben. Der große Vorteil einer durchgehenden Platte ist, man könnte die Elemente so anordnen, dass die Fläche von Leerräumen  minimiert wird. Die große Platte entfällt als Möglichkeit allerdings insofern, da diese nicht in der Schule produzierbar gewesen wäre.\\
    Eine andere Option wäre eine plattenlose, dabei würde man die Hutschienen direkt auf die Profilschienen des Serverschrankes montieren. Man spaart sich so eine Platte und die Elemente könnten direkt auf die Hutschienen montiert werden.
    Die plattenlose Option wäre eine kosteneffiziente Möglichkeit, allerdings gibt es viele Elemente, die im Schaltschrank nicht auf Hutschienen montiert werden können, diese bräuchten immer eine Montageplatte.\\
    Damit ein einheitliches Design eingehalten werden kann, wurde sich für ein Modulprinzip entschlossen. Dieses ermöglicht es allen Elementen, auch für die, die für Hutschienen ungeeignet sind, montiert zu werden und ist dennoch in der Schule produzierbar.

    \paragraph{Platten-Material}\mbox{}\\
    Für die Materialwahl gab es drei realistische Möglichkeiten. Die Modulplatten hätten vollständig aus Aluminium oder aus Dibond gefräst werden können. Als dritte Option hätte man die Platten aus einem Kunststoff lasern oder fräsen können. Die Aluplatten bieten den Vorteil der Leitfähigkeit und somit müsste man nur die Platte erden und die Elemente auf der Platte wären alle dementsprechend geerdet. Beim einer reinen Kunststoffplatte gibt es keine Leitfähigkeit und zusätzlich bieten die meisten Kunststoffe keine ausreichende mechanische Stabilität.\\
    \begin{wrapfigure}{r}{0.5\textwidth}
        \vspace{-30px}
        \includegraphics[width=0.5\textwidth]{Dibond_Platten_ml.jpg}
        \caption{Dibond-Platte, Quelle: \cite{Dibond-Platte}}
        \vspace{-20px}
        \label{fig:Sommerprototyp}
    \end{wrapfigure}    
    Aluminium erfüllt alle Anforderungen, ist aber teuer und ein wertvoller Werkstof. Da ein umsichtiger Umgang mit Ressourcen wichtig ist wurden nach einer Alternative gesucht. Dibond wurde daraufhin als Projektstandard für die Module definiert. Dieser Stoff besteht aus zwei dünnen Aluminiumplatten die auf einen Kunststoff aufgepresst werden. Dibond bietet keine elektrische Leitfähigkeit, folglich müssen alle Elemente zusätzlich geerdet werden aber das leichte Gewicht und die hohe mechanische Stabilität machen Dibond zur besten Option.

    \paragraph{Digitaler Zwilling}\mbox{}\\
    Moderner Schaltschrankherstellung begegnen im Herstellungsprozess oft große logistische Probleme. Jeder Prozessschritt ist eine Fehlerquelle und wenn Fehler nicht früh erkannt werden, pflanzen sich diese fort. Damit zwischen den Prozessschritten keine Kommunikationsprobleme entstehen setzen viele Hersteller auf das Prinzip des digitalen Zwillings.\\
    Dieser ist im Grunde ein digitaler Schaltschrank, welcher im ersten Prozessschritt, der Planung, ausgeplant wird und im Herstellungsprozess, sei es der Schrankbau oder die Bestückung, wird einerseits immer derselbe digitale Zwilling aktualisiert und aber auch referenziert. Das heißt alle Prozessschritte beziehen sich auf denselben Plan bzw. digitalen Zwilling (siehe \ref{fig:digilaerZwilling}).
    \begin{figure}[h]
        \centering
        \includegraphics[width=0.9\textwidth]{Cabinet-Building-Komax-SCB-Component-Printer.png}
        \caption{Digitaler Zwilling, Quelle: \cite{digitaler_zwilling_bild}}
        \label{fig:digilaerZwilling}
    \end{figure}
    Es setzt auch ein breites Feld an Firmen auf dieses Prinzip. Firmen wie Weidmüller, Komax, Steinhauer und noch viele mehr haben eine Firmenzusammenarbeit, die ohne einen digitalen Zwilling nicht möglich wäre \cite{smart_cabinet_building}. In diesem Fall werden die jeweiligen Prozessschritte meistens von einer neuen Firma übernommen. In diesem Bündnis ist der digitale Zwilling der Schlüssel zum Erfolg. Man kann dieses Prinzip der Dokumentation bzw. Planung als Industriestandard verstehen.\\
    Um den Prozess der Herstellung des Schaltschrankes möglichst nahe an die Praktiken aus der Industrie anzugleichen, wird auch der Schaltschrank des AFSS mithilfe eines digitalen Zwillings geplant. Dieser wird in Fusion360 gezeichnet und soll den Sollzustand des Schaltschrankes abbilden.\\
    Um die Konstruktion anzufangen, braucht es eine möglichst ausführliche Ausmessung des bereits bestehenden Serverschrankes. Besonders wichtig sind die Elemente die direkt am Umbau beteiligt sind, wie die Profilschienen (Abstände der Löcher, Abstände der Profilschienen zueinander und detaillierte Abmessungen der Profilschienen selbst.), die Türen und die Lüfter.\\

    \paragraph{Digitaler Zwilling - Umsetzung}\mbox{}\\ 

    \begin{wrapfigure}{r}{0.3\textwidth}
        \vspace{-20px}
        \includegraphics[width=0.3\textwidth]{Prototyp_Serverschrank.PNG}
        \caption{Sommerprototyp eines Serverschrankes}
        \vspace{-20px}
        \label{fig:Sommerprototyp}
    \end{wrapfigure}
    Als Erstes muss der Serverschrank, wie bereits erwähnt ausführlich ausgemessen werden. Äußere Höhe, innere Höhe, Äußere Breite, innere Breite und noch Vieles mehr muss richtig gemessen werden.    
    Bei den Messungen werden Messschieber und bei größeren Abständen Maßbänder verwendet. Um die mechanische Konstruktion zu erleichtern, werden alle Daten digital festgehalten.   
    Während die Messungen des Serverschranks für den finalen digitalen Zwilling wichtig werden, gibt es aber auch noch andere Punkte, beispielsweise bestand lange die Frage, ob das Modulkonzept so möglich sei. Aufgrund dessen und des Umfanges der Diplomarbeit sowie der begrenzten Zeit wird ein erster Entwurf eines Serverschrankes in Fusion360 konstruiert und weiters ein Probemodul gezeichnet. Die Maße dieses digitalen Prototyps sind von einem Standard-Serverschrank aus dem Internet übernommen.
    Dieser Prototyp hat nicht dieselben Werte wie der richtige Serverschrank, der dem AFSS zur Verfügung steht. Aufgrund der Prototyp-Konstruktion steht fest, dass das Modulkonzept ist so umsetzbar. Weiterführend ist festgestellt, wie man optimierter in Fusion Zeichnen kann.
    Eine Erkenntnis des Prototyps ist, dass man den Serverschrank nicht als ein großes Element konstruieren sollte, da wenn ein Fehler spät erkannt wird dieser so gut wie nicht mehr zu beheben ist. Wenn die Konstruktion allerdings auf viele verschiedene Elemente aufgeteilt wird, dann ist der Schaden bei einem Fehler begrenzt. 

    Nachdem das Grundprinzip erfolgreich konstruiert wurde, ist der Serverschrank auf Grundlage der echten Maße zu konstruieren. Die gelernten Erkenntnisse werden dabei bestmöglich miteinbezogen.\\    
    \begin{wrapfigure}{r}{0.35\textwidth} % 'r' for right, 'l' for left
        \vspace{-20px}
        \includegraphics[width=0.35\textwidth]{Schaltschrank_ohne_Module.PNG} 
        \caption{Serverschrank konstruiert}
        \vspace{-50px}
        \label{fig:Clean_Serverschrank}
    \end{wrapfigure}

    Mit dem fertig konstruierten Serverschrank, kann dann die Planung der Module beginnen. Dafür wird vorerst ein Konzept erstellt. Durchgedacht wird welche elektrischen Baugruppen wo im Schaltschrank platziert werden sollen. Das Konzept richtet sich einerseits nach der Vorgabe, zusammengehörige elektrische Komponenten sollen auf dasselbe Modul kommen, und andererseits zusammenhängende Module sollen sich möglichst nahe sein. 
    \paragraph{Modul 1 - Bedienelemente}\mbox{}\\
    Es sollen im Schaltschrank ein Notaus, ein Schlüsselschalter und ein Drehstromschalter verbaut werden. Diese Elemente werden gemeinsam auf einem Modul verbaut, da sie alle Bedienelemente sind.
    \paragraph{Modul 2 bis 3 - Schutzorgane und Versorgungen}\mbox{}\\
    Direkt unter dem Drehstromschalter sollen die Schutzorgane liegen. Da der Leitungsschutzschalter und der Fehlerstromschutzschalter wenig platz benötigen, werden auf dem 2. Modul zusätzlich die zwei Gleichrichter verbaut die auf eine Huttschiene montierbar sind. Damit besteht das 2. Modul aus den Schutzorganen und den Gleichrichtern. Auf die Platte muss damit auch eine durchgängige Huttschine und ein Verdrahtungskanal geplant werden.\\
    Für die reguläre Versorgung des AFSS werden drei normale Gleichrichter benötigt. Der dritte, der Deutronic Gleichrichter, kann nicht auf eine Hutschine montiert werden und hat zudem einen großen Platzbedarf. Dieser wird auf einem eigenen Modul verplant. In den Leerräumen des 3. Moduls werden Hutschinen mit Reihenklemmen verplant. Diese Reihenklemmen sollen für eine übersichtliche Verdrahtung der Versogungsleitungen sorgen.
    \paragraph{Modul 4 - ASI-Elemente}\mbox{}\\
    Konzepttechnisch ist das 4. Modul ein Erweiterungsmodul. Nur eine Ein/Ausgangsbaugruppe von Weidmüller ist auf eine kleine Huttschine verplant gewesen und der restliche Platz sollte freigelassen werden für potentielle Erweiterungen.\\
    Im Entwicklungsprozess ist deutlich geworden, dass die PWM-Signalerzeugung, der Ausgangsbaugruppen, nicht fähig sind ein veränderbares PWM-Signal zu erzeugen. Damit werden diese Elemnte nicht mehr benötigt. Parallel ist verstanden worden, dass es eine getrennte 24V-Versorgung für den ASI-Kreis geben muss. Deswegen ist im ursprünglichen Erweiterungsbereich dieses Modules, ein 24V-ASI-Gleichrichter verplant, der keine Huttschiene benötigt, und anstatt der Weidmüllerkomponenten kommmt eine ET200, an die ein ASI-Master angeschlossen ist. Zudem ist auch ein Verdrahtungskanal nötig.
    \paragraph{Modul 5 - SPS und Sicherungen}\mbox{}\\
    Das 5. Modul beinhaltet die Siemens SPS und die gesammten DC-Sicherungen für die Schrittmotoren. Für die SPS muss eine Siemens-Profilschiene auf die Platte und für die Sicherungen eine durchgängige Huttschiene und einen Verdrahtungskanal.
    \paragraph{Modul 6 - Schrittmotoren}\mbox{}\\
    Für die stärkeren Schrittmotoren werden Treiber verwendet die direkt auf einen Untergund montiert werden müssen. Die vierTreiber werden auf eine eigene Platte montiert. Auch auf diesem Modul ist ein Verdrahtungskanal nötig.
    \paragraph{Modul 7 - Ausgangsmodul}\mbox{}\\
    Am Ausgangsmodul soll einen Verdrahtungskanal haben und eine durchgängige Huttschiene. Auf diese Hutschine kommen Elemente wie die Motorschütz, Relais, die Treiber für die schwächeren Schrittmotoren und eine ausführliche Menge an Reihenklemmen.
    \paragraph{Modul 8 - Erdungsmodul}\mbox{}\\
    Etwaige Kabel des AFSS haben einen Schrim der geerdet gehört. Deswegen ist eine Erdungsplatte nötig die aus leitfähigen Aluminium gemacht werden soll. Auf dieser ist eine Ankerschiene zur Erdung von Leiterschirmen geplant. Zudem sollen in dieses Modul vier rechteckige Ausfräsungen gemacht werden, in welche die Buchse des RJ45-Stecker hineinpasst.
    \begin{wrapfigure}{r}{0.3\textwidth}
        \vspace{0px}
        \includegraphics[width=0.3\textwidth]{Vogis Bilder/SITOP_Konstruiert.png}
        \caption{Siemens SITOP in Fusion360}
        \vspace{-30px}
        \label{fig:SITOP_Konstrueiert}
    \end{wrapfigure}
    \paragraph{Fusion 360}\mbox{}\\
    Die einzelnden Module wurden, nachdem die Konzeptionierung beendet wurde, in Fusion360 gezeichnet. Dafür wurden zuerst die einzelnden Komponenten wie Hutschinen, Verdrahtungskanäle, Gleichrichter und alle anderen Komponenten als unabhängige Konstruktion gezeichnet. Damit wurde eine breite Bibliothek an Komponenten erstellt, mit dieser kann man dann wesentlich einfacher den digitalen Schaltschrank bestücken. Bei den jeweiligen Elementen ist es nicht wichtig kleinere Details miteinzubeziehen, vielmehr die Außenmase und die Positionen von Montagelöchern oder des Klemmmechanismuses für die Hutschiene sollen akkurat abgebildet sein (siehe \ref{fig:SITOP_Konstrueiert}).
    \begin{wrapfigure}{r}{0.4\textwidth}
        \vspace{0px}
        \includegraphics[width=0.4\textwidth]{Vogis Bilder/Bestuekter_Schaltschrank_Fusion.jpg}
        \caption{Der Schaltschrank in Fusion gezeichnet}
        \vspace{-30px}
        \label{fig:Schaltschrank_bestueckt_Fusion}
    \end{wrapfigure}
    \paragraph{Fusion 360}\mbox{}\\
    Nachdem man die einzelnen elektrischen Komponenten gezeichnet hat, wurde in Fusion die erste Platte gezeichnet. Hierbei muss man nicht direkt die richtige Größe einschätzen, da Fusion späte Änderungen gut zulässt. Nachdem man eine Platte im Serverschrank konstruiert hat, kann man die Hutschienen und Verdrahtungskanäle einfügen. Daraufhin kann man die Unterkonstruktion bestücken. Beim Einfügen von Konstruktionen in andere Konstruktionen bleiben die kopierten Elemente mit der originalen Zeichnung verknüpft. Diese Verknüpfung wurde bei den Elemten gelassen, die keine Änderungen erwarten doch bei Komponenten wie eines Verdrahtungskanals sollte man die Verknüpfung trennen um das Element zuschneiden zu können. Eine Verknüpfung bedeutet, dass Veränderungen nur in der ursprünglichen Zeichnung getätigt werden können. So wurde Modul nach Modul ausgeplant.\\
    Beim Zeichnen fiel auf, dass die Module aus dem Serverschrank, beziehungsweise Schaltschrank, rausragten. Es wurde erkannt, dass die Profilschienen nach innen verschoben werden müssen um zu ermöglichen, dass sich die Wände des Schaltschrankes auch schließen lassen. Die Bauart des Serverschrankes erlaubt eine solche Veränderung.\\ 
    Es wurden, bis auf das Erdungsmodul, alle Module auf diese Art konstruiert. Man konnte schön erkennen welche Reihenfolgen sinnvoll sind und wie sich welche Komponenten am besten montieren lassen können. Beispielsweise wurde ausgetestet ob bei gewissen elektrischen Komponenten eine vertikale Montage zielführend waren. In der Abbildung \ref{fig:Schaltschrank_bestueckt_Fusion} sieht man die fertige Konstruktion des Schaltschrankes.\\
    Zu dieser Version des Schaltschrankes ist zu sagen, dass ab der Realisierung des Geplanten, dieser digitale Plan sich fließend transformiert, zum digitalen Zwilling. Dieser Zwilling gehört nach der Realisierung konstant gepflegt. Das Erdungsmodul beispielsweise, welches erst mitten in der Realisierung eingeplant und gemacht wurde, muss in den digitalen Zwilling nachträglich hinzugefügt werden.
    \paragraph{AutoCAD}\mbox{}\\
    Nachem der digitale Plan/Zwilling vollständig ausgeplant wurde, mussten die jeweiligen Platten gefräst werden.Um die Platten auch fräsen zu können müssen diese zuerst als DXF-Datei in Filou-NC eingefügt werden. Dazu mussten die einzelnen Module in AutoCAD nachgezeichnet werden. Idealerweise hätte man hierfür nur die Oberfläche der Platten in Fusion zur Skizze gemacht und diese dann als DXF exportiert. Bei einem nachträglichen Ausmessen der 3D-Konstruktion wurde allerdings festgestellt, dass sich die Maße der Löcher der Profilschiene nicht mit der Realität decken. Zudem wurden weitere minimale Abweichungen festgestellt. Keiner der Fehler war so gravierend, dass man die Konstruktion des Schaltschrankes nicht weiterverwenden könnt aber für eine präzise Fräsung war die Fusion-Konstruktion nicht geeignet. Deswegen wurden sorgfälltig alle Module in AutoCAD nachgezeichnet.\\ 
    Beim Nachzeichnen wurde alle fehlerhaften Mase ausgebessert. Mit den ausgebesserten Mase wurde festgestellt, dass sich beim Modul 6 die vier Schrittmotortreiber nicht vertikal nebeneinander ausgehen. Deswegen wurde in AutoCAD einer der Treiber vertikal angeordnet.
    \paragraph{Räder}
\subsection{Realisierung}
\label{sec:Schaltplan}



