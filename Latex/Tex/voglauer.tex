\section{Elektroplanung und Realisierung \textcolor{gray}{(Nikolaj Voglauer)}}

\subsection{Elektroplanung}
\label{sec:Elektroplanung}

\subsubsection{Einleitung - Grundanforderungen}
    Wie werden die Anforderungen erfüllt, grob, Mobilität, Sicherheit, Modularität/Erweiterbarkeit, Produzierbarkeit.

\subsubsection{Elektrik spezififsche Anforderungen}
\label{sec:Elektrik spezififsche Anforderungen}

    \paragraph{Versogung}\mbox{}\\
    Hier schreibe ich darüber, dass die Motoren mit Gleichstrom betrieben werden. Wie viel Leistung die Motoren brauchen. Über die Sicherung jedes Einzelden. Weiters darüber, dass die Logikeinheiten - SPS - eine getrennte versorgung hat. Zudem auch die Versorgung der ASI Einheiten. 

    \paragraph{Ansteuerungen}\mbox{}\\
    Hier schreibe ich über wie die Schrittmotoren und wie sie angesteuert werden und abgesichert werden. Zudem Über den Asynchronmotor und wie er angesteuert wird und abgesichert. 

    \paragraph{Bedienelemente}\mbox{}\\
    Schlüsselschalter, Not-Aus, Dreiphasen-Drehschalter.

    \paragraph{Schaltschrank}\mbox{}\\
    Wie siehts aus mit dem Schaltschrank, warum kein neuer Schaltschrank und was muss der Server-Schrank erfüllen, damit er als Schaltschrank durchegeht. 

    \paragraph{Kabelauslegung}\mbox{}\\
    Welche Kabel müssen stark beweglich sein, welche Kabel müssen geschirmt werden, Querschnitte.

    \paragraph{Module}\mbox{}\\
    Erdungsanforderungen, Mechanische stabilität, reproduzierbarkeit

\subsubsection{Mechanische Planung}

    \paragraph{Modulprinzip}\mbox{}\\
    Hier wird über die Module geschrieben, warum sind diese sinnvoll und wieso wird nicht eine große Platte verwendet. Welche Materialen sind möglich und welche werden wieso verwendet und wie werden vom Gewählten Material die Anforderungen erfüllt.
    
    \paragraph{Digitaler Zwilling}\mbox{}\\
    Hier wird über das Prinzip eines digitalen Zwillings geschrieben und wieso dieser wichtig ist. "Um effektiv und recourcenschonend den Serverschrank umzubauen, ist es sinnvoll einen digitalen Zwilling..." 
    
    Wie fängt man einen digitalen Zwilling an?
    -Abmessungen: Alle Abmessungen die gemacht wurden anführen

    -Serverschrank in CAD Nachzeichnen: Dabei beschreiben was besonders wichtig ist, sowie das Bewegliche Teile in echt auch beweglich sein müssen in Fusion360. WIe waren die Schritte die beim Zeichnen gemacht wurden, mit was ich anfange.

    -Module: Ist der Schaltchrank fertig übernommen aus der realität kann man anfangen mit einem Modul. Dabei fällt einem zum Beispiel schon eine sache auf, sowie die Tatsache, dass wenn man die Profilschinen an der selben Position lässt würden sich die Türen nicht schließen lassen. Folge daraus ist, die Profilschinen müssen verschoben werden. Der Aufbau des Serverschranks lässt dies zu.

    -Weitere Module: diese werden dann auch in Fusion360 gezeichnet und dann in den digitalen Zwilling eingefügt. Dabei kann man schön erkennen wie die Module Anneinander gereiht gehören, welche Reihenfolge sinnvoll ist und ob sich die Menge an Elektrischen Einheiten auch ausgeht mit den Modulen. 

    -Dieser soll auch immer die akktuelle version abbilden


\subsection{Realisierung}
\label{sec:Schaltplan}



