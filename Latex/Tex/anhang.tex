\section{Anhang}

\subsection{Abmessungen (BSVN)}

\begin{figure}[H]
    \includegraphics[width=\textwidth]{abmessungen_rahmen.pdf}
    \centering
    \caption{Abmessungen Rahmen}
\end{figure}


\subsection{E-Plan}

\subsection{Projektmanagement (Witi)}
\textcolor{blue}{In diesem Kapitel soll auf das Projektmanagement des Projektes eingegangen werden. Zu Beginn empfiehlt es sich, die einzelnen Bereiche des Projektmanagements zu erklären und anschließend in einzelnen Kapiteln zu behandeln.}

\subsubsection{Aufgabenstellung des Gesamtprojekts}
\textcolor{blue}{Fügen Sie an dieser Stelle den Text der genehmigten Aufgabenstellung ein, der in die Diplomarbeitsdatenbank  eingegeben wurde.}

\subsubsection{Scrum-Projektplan}
\textcolor{blue}{Fügen Sie hier den vollständigen Scrum-Projektplan, wobei die Nummern der Tasks mit der Arbeitszeitaufzeichnung übereinstimmen müssen. Der Scrum-Projektplan kann auf mehrere Seiten aufgeteilt werden.}

\bgroup
    \centering
    \includegraphics[width=0.6\textwidth]{Scrum_Projektplan_mit_Tasks.png}
    \captionof{figure}{Scrum Projektplan mit Tasks}
\egroup

\newpage
\subsubsection{Terminplanung}

\newpage

\subsubsection{Arbeitspakete}

\paragraph{Maschinenbau (Simbürger)}
\begin{itemize}
    \item Konzeptionierung des Gesamtsystem
    \item CAD - Planung
    \item Komponentenfertigung
    \item Aufbau 
\end{itemize}

\paragraph{Softwareentwicklung (Simbürger)}
\begin{itemize}
    \item Benutzeroberfläche
    \item Warehouse Management System
    \item Datenbanken
\end{itemize}




\newpage

\subsection{Inbetriebnahme}

\subsubsection{Server}
Der Server kann durch zwei unterschiedliche Arten in Betrieb genommen werden.\\
Um beim Entwickeln einen angenehmeren Prozessablauf zu erlauben, kann der Server ganz klassisch durch Starten des Programms mit Python gestartet werden. Hierzu sollte zuerst eine neue Virtuelle Python umgebung aufgesetzt werden, und die Pakete aus dem "requirements.txt" geladen, oder die bei Entwickeln verwendete Virtuelle Umgebung in ".conda" Ordner Verwendet werden. Nach Aktivierung der Virtuellen Umgebung, kann durch ausführen des Befehls "python .\_\_init\_\_.py" im Ordner "Prototyp\_2" der Server gestartet werden. Im Terminal erscheinen dann die Logs des Einschaltvorgangs, wo auch die IP-Adresse und Port aufscheint, auf dem der Server läuft. Jedoch muss, um mit der Datenbank verbindung aufnehmen zu können, diese extra gestartet werden. Die Verbindungsdaten für die DB und viele andere Konfiguratonen können im "config.py" geändert werden.\\
Für einen Serverbetreb in einer Produktionsumgebung gibt es eigentlich die Möglichkeit den Server über einen Docker Container zu Starten. Nach installation von Docker, muss im Ordner über dem Projkt ("Prototyp\_2"), wo sich das File "docker-compose.yml" befindet, der Befehl "docker compose -f 'docker-compose.yml' up -d --build" ausgeführt werden. Danach wird ein Container erstellt. Diese kann dann entweder in Docker-Desktop oder über die Kommandozeile gestartet werden. Da die Datenbank beim ersten Starten auf einer Fremdmaschine jedoc noch kein Setup unterlaufen hat, ist es ratsam, diese mit einem save-file aus dem Ornder "DB\_export" aufzusetzen. Danach bleiben die Daten auch bei Abschaltung des Containers erhalten. Jedoch ist nach der Implementierung des Rust Pakets, das installieren der Libraries schwierig, und nicht im Compose enthalten.\\



\color{blue}
Nachdem typische Projekte aus mehreren Komponenten bestehen, ist es oft nicht trivial die einzelnen Komponenten korrekt zu konfigurieren und das Gesamtsystem in Betrieb zu nehmen. In diesem Kapitel soll eine vollständige, präzise und trotzdem möglichst kompak-te Anleitung zur Inbetriebnahme des Systems dargelegt werden. Die Schritte sollen in dem Detailgrad beschrieben werden, dass ein durchschnittlicher Schüler des vierten Jahrganges das Projekt in Betrieb nehmen kann. Exemplarisch sollten Punkte wie die folgenden be-handelt werden – die Aufzählung ist nicht vollständig):
\begin{itemize}
    \item Treiberinstallationen und Systemkonfigurationen
    \item Zu empfehlen wäre bei Server-Installationen ein Setup-Script, welches auf einem vordefinierten Docker-container aufbaut.
    \item Welche Schritte sind notwendig, um das Projekt mit dem vorhandenen Code / Schaltplänen (auf GIT, CD, Netzlaufwerk, etc.) in Betrieb zu nehmen.
    \item Bei Schaltungen mit mehreren Platinen muss beschrieben werden, wie diese mit-einander verbunden werden müssen.
\end{itemize}
\color{black}

\newpage
\subsection{Kostenaufstellung BSNV}
\textcolor{blue}{Für die Kalkulation im Gesamtprojekt sind folgende Kosten zu erfassen: \\
•	Kosten für Material (Hard- und Software)\\
•	externe Kosten (z.B.: Zukauf von Sensoren, Funkmodule, spezielle Entwicklungsum-gebungen, etc.) 
}
\begin{figure}[h]
    \includegraphics[width=0.8\textwidth]{Kostenaufstellung.png}
    \centering
    \caption{Kostenaufstellung}
\end{figure}

\newpage
\subsection{Besprechungsprotokolle (SV)}
%include pdf file as image, on howl page with label
\begin{figure}[H]
    \includegraphics[width=0.9\textwidth]{../Protokolls/Projektbesprechung_0.pdf}
    \centering
    \caption{Besprechungsprotokoll 10.12.2024}
\end{figure}

\begin{figure}[H]
    \includegraphics[width=0.9\textwidth]{../Protokolls/Projektbesprechung_1.pdf}
    \centering
    \caption{Besprechungsprotokoll 16.10.2024}
\end{figure}

\begin{figure}[H]
    \includegraphics[width=0.9\textwidth]{../Protokolls/Projektbesprechung_2.pdf}
    \centering
    \caption{Besprechungsprotokoll 10.12.2024}
\end{figure}

\begin{figure}[H]
    \includegraphics[width=0.9\textwidth]{../Protokolls/Projektbesprechung_3.pdf}
    \centering
    \caption{Besprechungsprotokoll x.x.xxxx}
\end{figure}




\newpage
\subsection{Arbeitsnachweis}

\subsubsection{Simbürger}

\begin{longtable}{|l|p{10cm}|r|}
    \hline
    \textbf{Datum} & \textbf{Tätigkeit} & \textbf{Stunden} \\
    \hline
    \endfirsthead

    \hline
    \textbf{Datum} & \textbf{Tätigkeit} & \textbf{Stunden} \\
    \hline
    \endhead

    \hline
    \endfoot

    \hline
    \endlastfoot

4.10.2023&Besprechung des weiteren vorgehens mit WB	&0.5\\
9.10.2023&Gruppenbesprechung für das weitere Vorgehen	&1.0\\
9.11.2023&Reconstruction Siemens Twin Towers	&3.5\\
14.11.2023&	Reconstruction Siemens Twin Towers	&2.0\\
23.11.2023&	Reconstruction Siemens Twin Towers Lasern	&2.0\\
29.11.2023&	RSTT Ausfahrer + Schlitten Fertigstellung&	4.0\\
19.12.2023	&CAD Auf/Ab-fahrer	&3.0\\
20.12.2023	&CAD RSTT, AutCad vorbereitung	&2.0\\

10.1.2024	&Testversuch ET200 SPS Stepdrive und Meeting Knapp&	4.0\\
11.1.2024	&Achse mechanisch fertig, Schlitten vertikal&	4.0\\
12.1.2024	&CAD Schuttel, Tag der offenen Tür vorbereitung	&2.0\\
16.1.2024	&Motoren ansteuern&	1.0	\\
20.1.2024	&Umlenkungen- und Aufhängungskonstruktion&	4.0\\
29.1.2024	&Diagramm Datenaustausch Anfertigung&   2.0\\
1.2.2024	&Besprechung WB&	4.0\\
3.2.2024	&Website Backend/Frontend Prototyp&	9.0\\
4.2.2024	&WS Frontend&	3.0\\

5.2.2024	&KWF Antrag schreiben und WS Datenbankmanagement&	3.0\\
7.2.2024	&OPC UA Client testung&	4.5\\
15.2.2024	&WS Suche usw, Organisation, Maschinenbaubesprechung	&5.5\\

20.2.2024	&Python / OPC UA Client testen	&1.0\\
23.2.2024	&WS Warenkorb, restructuring	&9.0\\
24.2.2024	&WS Warenkorb fertig, OPC anfang und Pflichtenheft Erstversion&	6.0\\
27.2.2024	&http-Kommunikation	testen&4.0\\
28.2.2024	&Lasten/Pflichtenheft erstellen	&3.0\\
4.3.2024	&http-Kommunikation testen&	3.5\\
10.3.2024	&CAD X-Achse& 7.0\\
13.3.2024	&Absprache mit WB bez. Pflichtenheft	&1.0\\
14.3.2024	&http-Kommunikation und CAD&	3.0\\
1.4.2024	&Datenbanken und Visualisierung	&5.0\\
2.4.2024	&CAD Lagerregal&	2.0\\
17.4.2024	&CAD Gabel und Software&	2.0\\
23.4.2024	&STT-Fortsetzung / Software einführung&	2.5\\
12.4.2024	&Datenbanken und Visu	&5.0\\
21.4.2024	&Datenbanken und Visu	&5.0\\
22.4.2024	&TF-IDF Recherche	&3.0\\
23.4.2024	&TF-IDF Implementierung	&2.0\\
28.4.2024	&Areas und Locations Implementierung	&6.0\\
19.4.2024	&Order Algorithmus konzeptionieren	&3.0\\
22.5.2024	&Order Algorithmus Implementierung	&3.0\\
2.6.2024	&Order Api Programmierung	&5.0\\
3.6.2024	&Api Implementierung und Visu& 3.0\\
4.6.2024	&STT-Fortsetzung und CAD&	2.0\\
6.6.2024	&SPS/Server Communictaion und Z-Prototyp CAD	&5.0\\
8.6.2024	&SPS Comm und Simulation implement	&4.0\\
9.6.2024	&System Controller	&4.0\\

14.6.2024	&Z-Prototyp Bauteile Vorbereitung& 1.0\\
18.6.2024	&Return, Cart programmieren	&4.0\\
19.6.2024	&Docker (f me)	&3.0\\
21.6.2024	&Z-PT, Schaltschrank, SPS-Com&	4.5\\
16.7.2024	&Recherche, Referenz-Elektronik&	1.0\\
17.7.2024	&Ref-Elektronik	&2.0\\
19.7.2024	&Designe/CAD Rollen u. Spannen y &6.0\\
22.7.2024	&Design X-Spannelement	&2.0\\
25.7.2024	&Design X-Spannelement und Rollen	&1.5\\
31.7.2024	&CAD Z-Achsen zauberei &	1.0\\
1.8.2024	&CAD Z-Achse redesign & 5.0\\
9.8.2024	&CAD YZ-Achse grobe fertigstellung&	5.0\\
10.8.2024	&CAD YZ-Achse feinerschliff	&4.0\\
11.8.2024	&CAD YZ-Achse + X-Achse beginn&	2.0\\
12.8.2024	&CAD X-Achse	&1.0\\
13.8.2024	&CAD X-Achse side roller	&4.0\\
14.8.2024	&CAD X-Achse side roller 2. side	&2.0\\
15.8.2024	&CAD X-Achse Mid rollers, side Wheels, YZ-Achse Spiegelung	&6.0\\
16.8.2024	&CAD YZ-Achse Lichttaster, X-Achse	&1.0\\
17.8.2024	&X-Achse Schleppkettengedanken Auslegung	&6.0\\
19.8.2024	&Schleppenderketten einplanung	&2.0\\
20.8.2024	&CAD vertikale Schleppkette &3.0\\
20.8.2024	&Sponsoren-E-mail beginn	&1.3\\
21.8.2024	&Kontaktdaten, Projektzusammenfassung	&1.5\\
22.8.2024	&CAD Schlitten Top 	&1.0\\
24.8.2024	&Stückliste, CAD Schlitten Top	&2.0\\
25.8.2024	&CAD X-Top Verbindung, Umlenkung	&5.0\\
27.8.2024	&CAD Rahmen Aufhängungen	&2.0\\
29.8.2024	&CAD Umlenkungen und Motoraufhängungen	&5.0\\
30.8.2024	&CAD Endschalter und Rahmen beginn	&3.0\\
31.8.2024	&CAD Rahmen, Lagerschrank beginn	&6.0\\
1.9.2024	&CAD Lagerschrank und Querfördererausschnitt	&2.0\\
2.9.2024	&Verbidungsslider implementieren	&1.0\\
3.9.2024	&Bugs beheben, Weidmüller Sortiment Bauteile auswählen	&4.0\\
4.9.2024	&Stack Bug behoben und CAD Querförderer	&7.0\\
5.9.2024	&CAD Mech. weitestgehende Fertigstellung	&4.0\\
9.9.2024	&Latex aufsetzen	&2.5\\
10.9.2024	&Meeting Weidmüller&	2.0	\\
11.9.2024	&Schaltschrank konzeptionieren &2.25\\
15.9.2024	&Stückliste anfertigen	&2.0\\

17.9.2024	&Autolager Demontieren für Bauteilbeschaffung&	2.5\\
18.9.2024	&Verbindungstest, Suchalgorythmus Rust implementation &6.0\\
19.9.2024	&Suchag. Fertig implementiert, CAD Rollen gezeichnet	&3.0\\

24.9.2024	&Projektmanagement&	3.0\\
25.9.2024	&Änderungen V-Slot-Aufhängung, Project-Libre	&1.0\\

26.9.2024	&Igus, Weidmüller, Motoren ansteuern die 1.&	4.0\\
28.9.2024	&Bux im Lageralgorithmus beheben	&1.5\\

1.10.2024	&CAD	&1.0\\
3.10.2024	&Profile bearbeiten	&1.5\\
8.10.2024	&Latex Vorlage	&1.0\\
9.10.2024	&Raumeinrichtung&	3.0\\
21.10.2024	&Rahmenbau beginn&	2.0\\
22.10.2024	&Rahmenbau und Drehstromverlegung, Auftragsvorbereitung X-Aufhängung &5.5\\
23.10.2024	&Rahmenbau	&3.0\\
24.10.2024	&Rahmenbau	&3.0\\
25.10.2024	&CAD XZ-Redesign	&3.0	\\
26.10.2024	&CAD XZ-Redesign	&7.0	\\
27.10.2024	&Auftragsvorbereitung X-Achse	&3.0\\
28.10.2024	&Auftragsvorbereitung X-Achse	&2.0	\\

5.11.2024	&Beginn Umlenkrollen Drehen	&0.8\\
6.11.2024	&Weidmüller DP Inbetrtiebnahme&	1.0\\
7.11.2024	&Weidmülller DP ansteuern&	3.5\\
8.11.2024	&Weidmülller DP ansteuern&	3.5\\
12.11.2024	&Umlenkrollen Drehen &	3.5\\
19.11.2024	&Umlenkrollen Drehen, Fräsen	&2.5\\
20.11.2024	&DAS: Drehen	&1.0\\
21.11.2024	&DAS: TFIDF	&2.0\\
22.11.2024	&DAS	&4.5\\
23.11.2024	&API stack , DAS	&3.0\\
24.11.2024	&DAS	&2.0\\
25.11.2024	&DAS	&1.5\\

26.11.2024	&Drehen& 3.5\\
29.11.2024	&DAS& 4.0\\
3.12.2024	&Drehen, DAS: Maschinenbau& 5.0\\
6.12.2024	&Website, DAS &3.5\\
10.12.2024	&DAS: Software, CAD	&5.0\\
13.12.2024	&Drehen &3.5\\
7.1.2025	&CNC-Fräsen, Hülsen Drehen&	3.5\\
8.1.2025	&X-Achse Zusammenbauen anfangen	&2.0\\
10.1.2025	&X-Achse Zusammenbauen	&4.0\\
14.1.2025	&X-Achse Zusammenbauen &	4.0\\
15.1.2025	&X-Achse Zusammenbauen &	4.5\\
17.1.2025	&Tag der offenen Türe	&5.0\\
21.1.2025	&TIA Portal Verbindung&	3.5\\
24.1.2025	&Lasern, Förderband, Mechanik	&3.5\\
4.2.2025	&Sicherheitstechnik-Besprechung, SPS - Server Kommunikation	&3.5\\
14.2.2025	&WMS Location Updateing	&1.0\\
17.2.2025	&DAS: Allgemeinteil	&1.0\\
21.2.2025	&Ref, Fräsen, Da-Schreiben	&5.0\\
25.2.2025	&Zusammenbauen&	4.0\\
28.2.2025	&Fräsen, E-Plan,&	3.5\\
2.3.2025	&DAS: Aufbau	&1.5\\
3.3.2025	&DAS: Besprechungsprotokolle&	1.0\\
5.3.2025	&Umlenkung und Motoren einbauen&	4.5\\
6.5.2025	&DAS  &	2.5\\
9.3.2025	&DAS XZ-Achse	&1.0\\
10.3.2025	&DAS	&3.0\\
11.3.2025	&DAS	&3.0\\
14.3.2025	&Verkabelung Schaltschrank	&4.0\\


    
\end{longtable}


\newpage
