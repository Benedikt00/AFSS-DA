\documentclass{article}
\pagenumbering{Roman}

\usepackage{ragged2e}
\usepackage[per-mode = fraction]{siunitx}
\usepackage{amsmath}
\usepackage{xcolor}
\usepackage{booktabs}
\usepackage{graphicx}
\usepackage[autostyle]{csquotes}
\usepackage{mathtools}
\usepackage[math-style=ISO, warnings-off={mathtools-colon, mathtools-overbracket}, ]{unicode-math}
\usepackage{polyglossia}
\setdefaultlanguage{english}
\setotherlanguage{german}
\setotherlanguage{greek}
\newfontfamily\greekfont{CMU Typewriter Text}
\usepackage[hidelinks]{hyperref}
%\setmainfont{CMU Typewriter Text}

\graphicspath{./photos/}
\author{Elena Widmann}
\title{first trys with \LaTeX}
\begin{document}

\NewDocumentCommand{\TheWriter}{}{The Girl}
\maketitle
\tableofcontents
\setlength{\parindent}{0pt}\newpage

\begin{abstract}
This paper will take you on a journey through the different stages of a story. But be careful, \TheWriter{} would be delighted to see you get confused and lost in between it's pages.
\end{abstract}


\newpage
\section{Beginnings}
It seems that this is how we start things.\\
We are off to a good start.
\section{Inbetweens}
How can I\slash You go to the Inbetween, if I\slash You are still stuck at the beginning?\\
Can you spot the difference between these words? \\You may have to look very carefully:\\
shuffling shuf\mbox{}fling shuf\mbox{}f\mbox{}ling
\section{Endings}
It seems that this is how we end things, \emph{however} we may feel about them.

\newpage 
Some endings are not infinite. On the contrary, they might continue after turning just one page. Are you ready for what the future might hold \ldots ?
\newpage\section*{A few thoughts from inside \TheWriter's head:}

\begin{german}
Sprichst du Deutsch?
\textenglish{This is a special message meant only for German speakers:}
Heute ist nicht der \today. (Der Tag existiert nämlich nicht.)
\end{german}\\[1cm]

\textlang{greek}{ἀλήθεια} is truth or disclosure in philosophy. Did you know this? It is important to me that you know this. Now you do. So don't ever forget, alright?\\[1cm]

\begin{Center}
    \ttfamily\label{this is me trying}\ldots They told me all of my cages were mental\\So I got wasted like all my potential\\And my words shoot to kill when I'm mad\\I have a lot of regrets about that\\I was so ahead of the curve, the curve became a sphere\\Fell behind on my classmates, and I ended up here\\Pouring out my heart to a stranger\\But I didn't pour the whiskey\\I just wanted you to know\\That this is me trying\ldots\footnote{Taylor Swift--This is me trying}\rmfamily\\[1cm]
\end{Center}

Every time \TheWriter{} goes outside, she does so with a hopeful feeling. She hopes that the brightness of the world will immediately give her a serotonin boost and that her problems will disappear into thin air. But soon she is reminded how much she hates the way her skin starts to burn and how exhausted she feels the moment she steps a foot into the heat. Everyday anew \TheWriter{} wishes she could simply enjoy the sun, but it seems that sometimes the brightest things are what push her to drown in the darkness the most.\\[1cm]

The key to live a happy life is to:
\begin{itemize}
    \item[-]Always look over your shoulder.
    \item[-]Never let your guard down.
    \item[-]Don't let people come too close.
    \item[-]Do none of the above!
\end{itemize} \newpage

The demons are called \enquote*{Nichevo'yas}. \TheWriter{} sleeps restless and screaming when they visit in her dreams. But when she wakes, panting and drenched in sweat, all she remembers are their blazing red eyes burning in the dark.\\[1cm]

\color[Gray]{7}\textquote[Lorde]{\color{cyan}Even when I was little, I knew that \textcolor{blue}{teenagers} sparkled. I knew they knew something \textcolor{blue}{children} didn't know, and \textcolor{blue}{adults} ended up forgetting.\color[Gray]{7}}\color{black}\\[1cm]

\begin{tabular}{cccc}
A & F & S & S \\
Automated & Factory & Storage & System \\
\label{Listing 1}
\end{tabular}\\[1cm]

\TheWriter{} can't draw, but I can! Have a look at what I did:\\
\begin{figure}[hb]
    \includegraphics[scale=0.4]{Shield2Go Verdrahtungsplan.png}
    \caption{My drawing (Not \TheWriter's)}
    \label{Figure}
\end{figure}\\[1cm]

\begin{Center}
    \Huge Do \normalsize you \Large ever \huge feel \tiny insignificant \large and \normalsize incredibly \Huge Huge \footnotesize at \Large the \small same \huge time?
\end{Center}

\newpage \section*{Maths is for freaks (numbers are cool though)}
\TheWriter's least favorite number is 3, she sees it everywhere. Three birds sitting on the fence outside her house, three eggs for breakfast or three black cats crossing the street on her way to the cemetery. Sometimes \TheWriter{} thinks it might be following her, she questions her own sanity. When you listen closely you can hear her mumbling to herself:
\[3+3=33+3=333+3=3333+3=33333+3=33\dots\]\\[1cm]

\NewDocumentCommand{\bAlpha}{}{\symbfup{\Alpha}}
\NewDocumentCommand{\bOmega}{}{\symbfup{\Omega}}
On her wall is a painted \textbf{Alpha} that \TheWriter{} stares at for hours. Nobody knows how it got there, she most certainly didn't paint it. Nevertheless, \TheWriter{} admires it's lively green color and the flawless brushstrokes a careful hand must have made. Despite countless hours of her watching the painted \(\bAlpha\), she never seems to notice the black and sloppy \(\bOmega\) right underneath it. Because there is no beginning without it's end.\\[1cm]

\begin{flushright}
    \footnote{Found on a piece of paper outside an old abandoned Asylum}
\end{flushright}
\[
\left]
\dfrac{3}{3 +
\dfrac{3}{3 +
\dfrac{3}{3 +
\dfrac{3}{3 +
\sqrt{3}}}}}
\right]
\]\\[1cm]

A recipe for nothing. (Trust me. You'll need it.):\\
\begin{itemize}
    \item[-] \qty{45}{\degreeCelsius} Air
    \item[-] \qty{153}{\l\per\m\squared} Invisibility
    \item[-] \qty{0}{VA} Apparent Power
    \item[-] \complexqty{3+2i}{\ohm} Resistance against the universe
\end{itemize} \newpage

\end{document}